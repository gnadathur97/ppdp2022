\section{Introduction}\label{sec:intro}

The Edinburgh Logical Framework, popularly referred to as LF, is a
typed lambda calculus in which types are permitted to depend on
terms. 
%
This dependency allows types to be used to encode relations between 
$\lambda$-terms, which have been argued to be good devices for
representing syntactic objects with binding
structure~\cite{miller87slp,pfenning88pldi}. 
%
Inhabitation of types, the content of typing judgements,
provides witnesses in this context to the validity of relevant
relations. 
%
Thus, LF constitutes an embodiment of the judgements-as-types principle
and it has been exploited in this capacity in the formalization of
rule-based presentations of varied object systems.

This paper is motivated by the overall goal of developing a framework
for stating and proving properties of formal specifications that are
constructed using LF.
%
Such a framework would provide a formal counterpart to the
informal style of reasoning about object systems based on their
rule-based description as is seen, for example, in \cite{pierce02book}.
%
Assuming a formalization using LF, a key part of such a framework
would be a logic that allows for the expression of complex properties
concerning LF typing judgements. 
%
Complementing such a logic, it would also be necessary to describe a
proof system that supports the mechanization of reasoning in the
logic, an aspect that would need to internalize case analysis and
induction over LF typing judgements. 

We deliver on the first of these requirements in this paper.
%
Specifically, we describe a logic, which we call \logic, in which
properties of LF specifications can be formally stated.
%
The logic is parameterized by an LF signature that describes the
particular object system that is of interest.
%
The atomic formulas in the logic then comprise typing judgements that
assert the inhabitation of types relative to LF contexts.
%
Complex formulas can be constructed from these judgements using a
collection of propositional constants and connectives and
quantifiers.
%
The quantification that is permitted ranges over expressions that
denote both LF terms and LF contexts.
%
A critical issue that must be addressed in this context is the
qualification of quantifiers to limit their scope to meaningful
subclasses of expressions.
%
For term quantifiers, we address this issue by using a special class
of simple types that we call \emph{arity types}.
%
Such a type limits attention to terms that satisfy a particular functional
structure, leaving deeper, dependency related properties to the domain
of the formula that the quantifier ranges over.
%
For context quantifiers, we introduce a new kind of types that we call
\emph{context schemas}.
%
These types, which are motivated by the \emph{regular worlds} used
with Twelf developments~\cite{Pfenning02guide,schurmann00phd},
identify patterns of declarations out of which the actual contexts
that instantiate the quantifiers must be constructed.
%
The validity of atomic formulas that are devoid of free variables is
determined, as might be expected, by the derivability of the LF
judgements that they represent.
%
Propositional symbols are understood in the usual manner.
%
An especially interesting part of the semantics is the treatment of
quantifiers: these are interpreted via the substitution of expressions
that adhere to the type qualifications that adorn them.

We have also developed a proof system that provides the basis
for mechanizing reasoning relative to \logic. 
%
This proof system is oriented around sequents that intuitively
encapsulate states in the development of a proof.
%
The truly innovative part of the system is its treatment of atomic
formulas that embodies their interpretation as typing judgements in LF.
%
Included in this part is a mechanism for analyzing
an atomic assumption formula via the parameterizing signature and the
declarations in the LF context that is a part of the formula.
%
The rules also build in the capability to reason by induction on the
heights of LF derivations and to utilize meta-theorems about LF
derivability.
%
While the detailed presentation of this proof system is beyond the scope of
this paper, we provide a sketch of it for the sake of completeness. 
%
A complete development may be found in \cite{nadathur21arxiv,southern21phd}.
%
We also note that an implementation of the framework is available in
the Adelfa system \cite{adelfa.website,southern21lfmtp}.

The rest of the paper is organized as follows.
%
In the next section, we present the version of LF that is the basis of
this work, illustrate its use in formalizing object systems and
identify the kinds of properties that we might want to represent
relative to such formalizations. 
%
We also introduce here the notion of  \emph{simultaneous
hereditary substitution} that generalizes ideas in~\cite{harper07jfp}
and that plays a significant role in the development of our logic. 
%
Section~\ref{sec:logic} then presents the logic \logic\ and illustrates
its use in formally describing properties of LF specifications. 
%
Section~\ref{sec:proofsystem} sketches a proof system that provides
the means for mechanizing reasoning relative to the logic. 
%
%% Section~\ref{sec:examples} provides a hint of how such a proof system
%% might work in the formalization of arguments of validity. 
%% %
We conclude the paper with a brief discussion of other work related to
this topic. 
