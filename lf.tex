\section{LF and its Use in Formalizations}
\label{sec:lf}

We use in this work the presentation of LF that is referred to as
\emph{canonical LF}.
%
This version of LF admits only terms that are in $\beta$-normal form
and that are well-typed only if they are additionally in $\eta$-long
form~\cite{harper07jfp,watkins03tr}.
%
An important part of such a treatment is a definition of
substitution---referred to as \emph{hereditary substitution}---that
builds in $\beta$-reduction.

The first subsection below presents canonical LF.
%
A distinctive aspect of our presentation is a treatment of
substitution that is independent of LF typing judgements and its
extension to include the simultaneous replacement of multiple
variables.
%
These generalizations, that are developed more completely
in~\cite{nadathur21arxiv}, are important to the logic and the proof
system that we describe in later sections. 
%
Towards motivating the development of the logic, 
we then discuss the use of LF in representing
object systems and in reasoning about them at an informal level.
%
The section concludes with an identification of meta-theorems related
to derivability in LF that will be used in articulating proof rules in
our logic.

\subsection{Canonical LF}\label{ssec:lf-syntax}

We describe, in turn, the syntax of canonical LF, the notion of
substitution and the wellformedness judgements below. 

\begin{figure}[htpb]
\[
\begin{array}{r r c l}
  \mbox{\bf Kinds} & K & ::= & \type\ |\ \typedpi{x}{A}{K}\\[5pt]

  \mbox{\bf Canonical Type Families} & A,B & ::= &
           P\ |\ \typedpi{x}{A}{B}\\
  \mbox{\bf Atomic Type Families} & P & ::= & a\ |\ P\app M\\[5pt]
  \mbox{\bf Canonical Terms} & M,N & ::= & R\ |\ \lflam{x}{M}\\
  \mbox{\bf Atomic Terms} & R & ::= & c\ |\ x\ |\ R\app M\\[5pt]
  \mbox{\bf Signatures} & \Sigma & ::= &
  \emptysig\ |\ \Sigma,c:A\ |\ \Sigma,a:K\\[5pt]
  \mbox{\bf Contexts} & \Gamma & ::= & \emptyctx\ |\ \Gamma,x:A
\end{array}
\]
\caption{The Syntax of LF Expressions}
\label{fig:lf-terms}
\end{figure}

\subsubsection{The Syntax} The syntax of LF expressions is described in
Figure~\ref{fig:lf-terms}.
%
The primary interest is in three categories of expressions:
kinds, types which are indexed by kinds, and terms which are indexed
by types. 
%
In these expressions, $\lambda$ and $\Pi$ are binding or abstraction
operators.
%
Relative to these operators, we assume the principle of
equivalence under renaming that is applied as needed. 
%
We also assume as understood the notions of free and bound
variables that are usual to expressions involving such operators.
%
To ensure the absence of $\beta$-redexes, terms are stratified into
\emph{canonical} and \emph{atomic} forms.  
%
A similar stratification is used with types that is exploited by the
formation rules to force all well-typed terms to be in $\eta$-long
form.
%
We use $x$ and $y$ to represent term-level variables, which are bound
by abstraction operators or in the contexts that are associated with
terms. 
%
Further, we use $c$ and $d$ for term-level constants, and $a$
and $b$ for type-level constants, both of which are typed in
signatures. 
%
The expression $\imp{A_1}{A_2}$ is used as an alternative notation
for the type family $\typedpi{x}{A_1}{A_2}$ when $x$ does
not appear free in $A_2$. 
%
An atomic term has the form $(h\app M_1\app \ldots\app M_n)$ where $h$
is a variable or a constant.
%
We refer to $h$ as the \emph{head symbol} of such a term. 

\subsubsection{Simultaneous Hereditary Substitution}

To preserve the form of LF expressions, it is necessary to build 
$\beta$-reduction into the application of substitutions.
%
An important consideration in this context is that substitution
application must be a terminating operation. 
%
Towards ensuring this property, substitutions are indexed by types
that are eventually intended to characterize the functional structure
of expressions. 

\begin{definition}\label{def:aritytypes}
The collection of expressions that are obtained from the constant
$\oty$ using the binary infix constructor $\atyarr$ constitute the
{\it arity types}.
%
Corresponding to each canonical type $A$, there is an arity type
called its {\it erased form} and denoted by $\erase{A}$ and given as
follows: $\erase{P} = \oty$ and $\erase{\typedpi{x}{A_1}{A_2}} =
\arr{\erase{A_1}}{\erase{A_2}}$.  

\end{definition}

\begin{definition}\label{def:substitution}
A substitution $\theta$ is a finite set of 
the form
\[\{\langle x_1,M_1,\alpha_1 \rangle, \ldots, 
    \langle x_n, M_n, \alpha_n \rangle \},\]
where, for $1 \leq i \leq n$, $x_i$ is a
distinct variable, $M_i$ is a canonical term and $\alpha_i$ is an
arity type.\footnote{Note that by a 
systematic abuse of notation, $n$ may be less than $m$ in a sequence written in
the form $s_m,\ldots,s_n$, in which case the empty sequence is
denoted. In this particular instance, a substitution can be an empty
set of triples.}
%
Given such a substitution, $\domain{\theta}$ denotes the set 
$\{x_1,\ldots,x_n\}$ and $\range{\theta}$ denotes the set
$\{M_1,\ldots,M_n\}$.
\end{definition}

\begin{figure*}[tbhp]

\begin{tabular}{lccc}    
\raisebox{2.4ex}{\fbox{$\hsub{\theta}{M}{M'}$}}\hspace{0.5cm} &
  
  \infer{\hsub{\theta}
              {R}
              {R'}}
        {\hsubr{\theta}
              {R}
              {R'}} \qquad
  & \qquad
  \infer{\hsub{\theta}
              {R}
              {M'}}
        {\hsubr{\theta}
              {R}
              {M':\alpha'}}

  & \qquad
\infer{\hsub{\theta}
            {(\lflam{x}{M})}
            {\lflam{x}{M'}}}
      {x\ \mbox{\rm not free in}\ \domain{\theta} \cup \range{\theta}
        \qquad
        \hsub{\theta}{M}{M'}}
\\[10pt]


\raisebox{2.4ex}{\fbox{$\hsubr{\theta}{R}{M':\alpha'}$}}\hspace{0.5cm} &
  
\infer{\hsubr{\theta}{x}{M:\alpha}}
      {\langle x,M,\alpha \rangle \in \theta}

& 

\multicolumn{2}{c}
            {\qquad \infer{\hsubr{\theta}{(R\app M)}{M''':\alpha''}}
                   {\hsubr{\theta}{R}{\lflam{x}{M'}:\arr{\alpha'}{\alpha''}} 
                      \qquad
                    \hsub{\theta}{M}{M''}
                      \qquad
                    \hsub{\{\langle x, M'',\alpha'\rangle \}}{M'}{M'''}}}

\\[10pt]

\raisebox{2.4ex}{\fbox{$\hsubr{\theta}{R}{R'}$}}\hspace{0.5cm} &
  
\infer{\hsubr{\theta}{c}{c}}
      { }
& \qquad
\infer{\hsubr{\theta}
            {x}
            {x}}
      {x\not\in\domain{\theta}}

& \qquad 
\infer{\hsubr{\theta}{(R\app M)}{R'\app M'}}
       {\hsubr{\theta}{R}{R'} \qquad\qquad
        \hsub{\theta}{M}{M'}}
\end{tabular}
\caption{Applying Substitutions to Terms}
\label{fig:hsub}
\end{figure*}

Given a substitution $\theta$ and an expression $E$ that is a kind, a
type, or a canonical term, we wish the expression
$\hsubst{\theta}{E}$ notionally to denote the application of $\theta$
to $E$.
%% %
%% Given a substitution $\theta$ and an expression $E$ that is a kind, a
%% type, a canonical term or a context, we wish the expression
%% $\hsubst{\theta}{E}$ notionally to denote the application of $\theta$
%% to $E$.
%% %
However, such an application is not guaranteed to exist.
%
We therefore use the expression $\hsub{\theta}{E}{E'}$ to indicate
when it is defined and has $E'$ as a result.
%
The key part of defining this relation is that of articulating
its meaning when $E$ is a canonical term.
%
This is done in Figure~\ref{fig:hsub} via rules for deriving this
relation. 
%
These rules use an auxiliary definition of substitution into an
atomic term which accounts for any normalization that is necessitated
by the replacement of a variable by a term.
%
The different categories of rules in this figure are distinguished by
being preceded by a box containing the judgement form they relate to.
%
The extension of this definition to the case where $E$ is a kind or a type
corresponds essentially to the application of the substitution to the
terms that appear within $E$. 
%
This idea is made explicit for types in Figure~\ref{fig:hsubtypes} and its
elaboration for kinds is similar.
%
%% A substitution is meaningfully applied to a context only when it does
%% not replace variables to which the context assigns types and when a
%% replacement does not lead to inadvertent capture.
%% %
%% When these conditions are satisfied, the substitution distributes to
%% the types that are assigned to the variables as the rules in
%% Figure~\ref{fig:hsubctx} make clear. 


\begin{figure}[tbhp]
\begin{center}
\begin{tabular}{c}
\infer{\hsub{\theta}{a}{a}}
      { }

\qquad \qquad
  
\infer{\hsub{\theta}{(P\app M)}{(P'\app M')}}
      {\hsub{\theta}{P}{P'} &
       \hsub{\theta}{M}{M'}}

\\[10pt]

\infer{\hsub{\theta}{(\typedpi{x}{A_1}{A_2})}{\typedpi{x}{A_1'}{A_2'}}}
      {x\ \mbox{\rm not free in}\ \domain{\theta} \cup \range{\theta}
        \qquad
        \hsub{\theta}{A_1}{A_1'} \qquad
       \hsub{\theta}{A_2}{A_2'}}
\end{tabular}
\end{center}

\caption{Applying Substitutions to Types}
\label{fig:hsubtypes}
\end{figure}

%% \begin{figure}[tbhp]
%% \begin{tabular}{c}
%%   \infer
%%     {\hsub{\theta}
%%           {\emptyctx}
%%           {\emptyctx}}
%%     {}

%% \\[10pt]
    
%%   \infer
%%     {\hsub{\theta}{(\Gamma, x:A)}{\Gamma', x:A'}}
%%     {x\ \mbox{\rm not free in}\ \domain{\theta} \cup \range{\theta} \qquad
%%      \hsub{\theta}{\Gamma}{\Gamma'} \qquad
%%      \hsub{\theta}{A}{A'}} 
%% \end{tabular}
%% \caption{Applying Substitutions to Contexts}
%% \label{fig:hsubctx}
%% \end{figure}

We define a measure on substitutions that is useful in showing that
their application terminates. 

\begin{definition}\label{def:typesubsize}
The size of an arity type is the number of occurrences of $\atyarr$ in
it. The size of a substitution is the largest of the sizes of the arity types
in each of its triples.
\end{definition}

The following theorem can be proved by induction first on
the sizes of substitutions and then on the structures of expressions;
it would first be proved simultaneously for canonical and atomic terms
and then extended to types and kinds. 
%
%% The following theorem can be proved by induction first on
%% the sizes of substitutions and then on the structures of expressions;
%% it would first be proved simultaneously for canonical and atomic terms
%% and then extended to types, kinds and contexts. 
%% %
\begin{theorem}\label{th:uniqueness}
For any kind, type or canonical term $E$ and any substitution
$\theta$, it is decidable whether there is an $E'$ such that
$\hsub{\theta}{E}{E'}$ is derivable.
%
%% For any context, kind, type or canonical term $E$ and any substitution
%% $\theta$, it is decidable whether there is an $E'$ such that
%% $\hsub{\theta}{E}{E'}$ is derivable.
%% %
Moreover, there is at most one $E'$ for which it is derivable.
%
Similarly, for any atomic term $R$ and substitution $\theta$, it is
decidable whether there is an $R'$ or an $M'$ and $\alpha'$ such
that $\hsubr{\theta}{R}{R'}$ or $\hsubr{\theta}{R}{M' : \alpha'}$ is
derivable, at most one of these judgements is derivable and that too
for a unique $R'$, respectively, $M'$ and $\alpha'$. 
\end{theorem}

% The following property has an obvious proof by induction on the
% structure of the expression.

% \begin{theorem}\label{th:vacuoussubs}
% If $E$ is a kind, a type or a canonical term none of whose free
% variables is a member of $\domain{\theta}$, then $\hsub{\theta}{E}{E}$ has
% as derivation. If $R$ is an atomic term none of whose free variables
% is a member of $\domain{\theta}$ then $\hsubr{\theta}{R}{R}$ has a
% derivation.
% \end{theorem}

% Simultaneous hereditary substitution enjoys a permutation 
% property that is stated in the following theorem. This property is
% similar to the one described in \cite{harper07jfp} for unitary
% substitution and its proof is similar.

% \begin{theorem}\label{th:subspermute}
% Let $\theta_1$ be the substitution
% \[\{\langle x_1, M_1,\alpha_1
% \rangle, \ldots \langle x_n,M_n,\alpha_n \rangle \}.\]
% %
% Further, let
% $\theta_2$ be the substitution  $\{\langle y_1, N_1,\beta_1
% \rangle, \ldots \langle y_m,N_m,\beta_m \rangle \}$ where
% $y_1,\ldots,y_m$ are variables that are distinct from $x_1,\ldots,x_n$ 
% and that do not appear free in $M_1,\ldots,M_n$.  
% %
% Finally, suppose that for each $i$, $1 \leq i \leq m$, there is some
% $N'_i$ such that $\hsub{\theta_1}{N_i}{N'_i}$ has a derivation and let
% $\theta_3 = \{ \langle y_1, N'_1,\beta_1 \rangle, \ldots \langle
% y_m,N'_m,\beta_m \rangle \}$.
% %
% For every kind, type and canonical term $E$, $E_1$ and
% $E_2$ such that $\hsub{\theta_1}{E}{E_1}$ and $\hsub{\theta_2}{E}{E_2}$ have
% derivations, there must be an $E'$ such 
% that $\hsub{\theta_1}{E_2}{E'}$ and $\hsub{\theta_3}{E_1}{E'}$ have
% derivations.
% \end{theorem}

\begin{figure}[tbhp]
\begin{center}

  \begin{tabular}{c}

  \infer
      {\stlctyjudgr{\STLCGamma}{c}{\alpha}}
      {c: \alpha \in \STLCGamma}

  \qquad
  \infer
      {\stlctyjudgr{\STLCGamma}{x}{\alpha}}
      {x: \alpha \in \STLCGamma}

  \qquad
  \infer
      {\stlctyjudgr{\STLCGamma}{R \app M}{\alpha}}
      {\stlctyjudgr{\STLCGamma}{R}{\alpha' \atyarr \alpha} \qquad
       \stlctyjudg{\STLCGamma}{M}{\alpha'}}

  \\[15pt]
  \infer
      {\stlctyjudg{\STLCGamma}{\lflam{x}{M}}{\alpha_1 \atyarr \alpha_2}}
      {\stlctyjudg{\aritysum{\{x:\alpha_1\}}{\STLCGamma}}{M}{\alpha_2}}

  \qquad
  \infer
      {\stlctyjudg{\STLCGamma}{R}{\oty}}
      {\stlctyjudgr{\STLCGamma}{R}{\oty}}
  \end{tabular}
\end{center}
\caption{Arity Typing for Canonical Terms}\label{fig:aritytyping}
\end{figure}

While the application of a substitution to an LF expression may not
always exist, this is guaranteed to be the case when certain arity
typing constraints are satisfied as we describe below.

\begin{definition}\label{def:aritytyping}
\begin{enumerate}
\item An \emph{arity context} $\STLCGamma$ is a set of unique
  assignments of arity types to (term) constants and variables; these
  assignments are written as $x:\alpha$ or $c:\alpha$.
%
\item Given two arity contexts $\STLCGamma_1$ and $\STLCGamma_2$, we write
$\aritysum{\STLCGamma_1}{\STLCGamma_2}$ to denote the collection of all the 
assignments in $\STLCGamma_1$ and the assignments in $\STLCGamma_2$ to
the constants or variables not already assigned a type in $\STLCGamma_1$. 
%
\item
The rules in Figure~\ref{fig:aritytyping} define the arity typing
relation denoted by $\stlctyjudg{\STLCGamma}{M}{\alpha}$ between
a term $M$ and an arity type $\alpha$ relative to an arity context
$\STLCGamma$. 
%
\item 
A kind or type $E$ is said to respect an arity context $\STLCGamma$
under the following conditions: if $E$ is $\type$; if $E$ is an atomic
type and for each canonical term $M$ appearing in $E$ there is an
arity type $\alpha$ such that $\stlctyjudg{\STLCGamma}{M}{\alpha}$ is 
derivable; and if $E$ has the form $\typedpi{x}{A}{E'}$ and $A$
respects $\STLCGamma$ and $E'$ respects
$\aritysum{\{x:\erase{A}\}}{\STLCGamma}$. 
%
A context $\Gamma$ is said to respect $\STLCGamma$ if
for every $x:A$ appearing in $\Gamma$ it is the case that $A$ respects
$\STLCGamma$.
%
\item
A substitution $\theta$ is {\it arity type preserving}
with respect to $\Theta$ if for every $\langle x,M,\alpha \rangle \in
\theta$ it is  the case that $\stlctyjudg{\STLCGamma}{M}{\alpha}$ is
derivable. 
%
\item
Given a substitution $\theta$, the expression $\context{\theta}$
denotes the arity context $\{ x : \alpha\ \vert\ \langle x, M, \alpha
\rangle \in \theta \}$. 
%
\end{enumerate}
\end{definition}

\begin{theorem}\label{th:aritysubs}
Let $\theta$ be a substitution that is arity type preserving with
respect to $\STLCGamma$ and let $\STLCGamma'$ denote the arity context 
$\aritysum{\context{\theta}}{\STLCGamma}$. 
%
\begin{enumerate}
\item If $E$ is a canonical type or kind that respects the
  arity context $\STLCGamma'$, then there must be an $E'$ that
  respects $\STLCGamma$ and that is such that $\hsub{\theta}{E}{E'}$
  is derivable. 

\item If $M$ is a canonical term such that
$\stlctyjudg{\STLCGamma'}{M}{\alpha}$ is
derivable, then there must be an $M'$ such that there are derivations
for $\hsub{\theta}{M}{M'}$ and $\stlctyjudg{\STLCGamma}{M'}{\alpha}$.

\item If $R$ is an atomic term such that
  $\stlctyjudgr{\STLCGamma'}{R}{\alpha}$
  is derivable, then either there is an atomic term $R'$ such that
  $\hsub{\theta}{R}{R'}$ and $\stlctyjudgr{\STLCGamma}{R}{\alpha}$ are
  derivable or there is a canonical term $M$ such that
  $\hsub{\theta}{R}{M : \alpha}$ and
  $\stlctyjudg{\STLCGamma}{M}{\alpha}$ are derivable. 
\end{enumerate}
\end{theorem}
\begin{proof}
The first clause in the theorem is an easy consequence of the
second. Clauses (2) and (3) are proved simultaneously by induction
first on the sizes of substitutions and then on the structure of
terms.
% 
The argument proceeds by considering the cases for the term structure,
first proving (3) and then using this in proving (2). 
\end{proof}

We will often consider expressions and substitutions that satisfy the
arity typing requirements of the theorem above, which then guarantees
that the applications of the substitutions have results.  
%
We introduce a notation that is convenient in this situation: we will
write $\hsubst{\theta}{E}$ to denote the unique $E'$ such that
$\hsub{\theta}{E}{E'}$ has a derivation whenever such a derivation is
known to exist.


The erased form of a type is invariant under substitution. This is the
content of the theorem below whose proof is straightforward.

\begin{theorem}\label{th:erasure}
For any type $A$ and substitution $\theta$, if $\hsub{\theta}{A}{A'}$
has a derivation, then $\erase{A} = \erase{A'}$.
\end{theorem}

\subsubsection{Wellformedness Judgements}

\begin{figure*}[htbp]

\begin{tabular}{lc}
\raisebox{5.5ex}{\fbox{$\lfsig{\Sigma}$}}

& 
\begin{tabular}{c}

\infer[\sigempty]{\lfsig{\emptysig}}{} \qquad
  
\infer[\sigterm]
      {\lfsig{\Sigma,c:A}}
      {\lfsig{\Sigma} \qquad \lftype{\emptyctx}{A} \qquad c\ \mbox{\rm does not
          appear in}\ \Sigma}

\\[10pt]

\infer[\sigfam]
      {\lfsig{\Sigma,a:K}}
      {\lfsig{\Sigma} \qquad \lfkind{\emptyctx}{K} \qquad a\ \mbox{\rm does not
          appear in}\ \Sigma}
\end{tabular}

\\[40pt]

\raisebox{1.3ex}{\fbox{$\lfctx{\Gamma}$}} 

& 
\begin{tabular}{c}
    
\infer[\ctxempty]
      {\lfctx{\emptyctx}}{}

\qquad
      
\infer[\ctxterm]
      {\lfctx{\Gamma,x:A}}
      {\lfctx{\Gamma} \qquad \lftype{\Gamma}{A} \qquad x\ \mbox{\rm does not
          appear free in}\ \Gamma}
\end{tabular}

\\[20pt]

\raisebox{1.3ex}{\fbox{$\lfkind{\Gamma}{K}$}}

& 

\begin{tabular}{c}
\infer[\canonkindtype]
      {\lfkind{\Gamma}{\type}}{}

\qquad 
\infer[\canonkindpi]
      {\lfkind{\Gamma}{\typedpi{x}{A}{K}}}
      {\lftype{\Gamma}{A} \qquad
       \lfkind{\Gamma,x:A}{K}}
\end{tabular}

\\[20pt]

\raisebox{1.3ex}{\fbox{$\lftype{\Gamma}{A}$}}

& 

\begin{tabular}{cc}       
\infer[\canonfamatom]
      {\lftype{\Gamma}{P}}
      {\lfsynthkind{\Gamma}{P}{\type}}\quad & \quad
\infer[\canonfampi]
      {\lftype{\Gamma}{\typedpi{x}{A_1}{A_2}}}
      {\lftype{\Gamma}{A_1} \qquad \lftype{\Gamma, x:A_1}{A_2}}
\end{tabular}

\\[20pt]

\raisebox{1.3ex}{\fbox{$\lfsynthkind{\Gamma}{P}{K}$}}

& 

\begin{tabular}{cc}
\infer[\atomfamconst]
      {\lfsynthkind{\Gamma}{a}{K}}
      {a:K\in\Sigma}
&
\infer[\atomfamapp]
      {\lfsynthkind{\Gamma}{P\app M}{K}}
      {\lfsynthkind{\Gamma}{P}{\typedpi{x}{A}{K_1}} \qquad
       \lfchecktype{\Gamma}{M}{A} \qquad
      \hsub{\{\langle x, M, \erase{A}\rangle\}}{K_1}{K}}
\end{tabular}

\\[20pt]

\raisebox{1.3ex}{\fbox{$\lfchecktype{\Gamma}{M}{A}$}}

&

\begin{tabular}{cc}
\infer[\canontermatom]
      {\lfchecktype{\Gamma}{R}{P}}
      {\lfsynthtype{\Gamma}{R}{P}} \quad & \quad 
\infer[\canontermlam]
      {\lfchecktype{\Gamma}{\lflam{x}{M}}{\typedpi{x}{A_1}{A_2}}}
      {\lfchecktype{\Gamma,x:A_1}{M}{A_2}}
\end{tabular}

\\[20pt]

\raisebox{5.5ex}{\fbox{$\lfsynthtype{\Gamma}{R}{A}$}}

&

\begin{tabular}{c}
\infer[\atomtermvar]
      {\lfsynthtype{\Gamma}{x}{A}}
      {x:A\in\Gamma}
\qquad
\infer[\atomtermconst]
      {\lfsynthtype{\Gamma}{c}{A}}
      {c:A\in\Sigma}
      
\\[10pt]
      
\infer[\atomtermapp]
      {\lfsynthtype{\Gamma}{R\app M}{A}}
      {\lfsynthtype{\Gamma}{R}{\typedpi{x}{A_1}{A_2}} \qquad
       \lfchecktype{\Gamma}{M}{A_1} \qquad
       \hsub{\{\langle x, M, \erase{A_1}\rangle\}}{A_2}{A}} 
 \end{tabular}

\end{tabular}

\caption{The Formation Rules for LF}
\label{fig:lf-judgements}
\end{figure*}

Canonical LF includes seven judgements: $\lfsig{\Sigma}$ that ensures
that the constants declared in a signature are distinct and their type
or kind classifiers are well-formed; $\lfctx{\Gamma}$ that ensure that
the variables declared in a signature are distinct and their type
classifiers are well-formed in the preceding declarations and
well-formed signature $\Sigma$; $\lfkind{\Gamma}{K}$ that determines
that a kind $K$ is well-formed with respect to a well-formed signature
and context pair; $\lftype{\Gamma}{A}$ and
$\lfsynthkind{\Gamma}{P}{K}$ that check, respectively, the formation
of a canonical and atomic type relative to a well-formed 
signature, context and kind triple; and $\lfchecktype{\Gamma}{M}{A}$ and
$\lfsynthtype{\Gamma}{R}{A}$ that ensure, respectively, that a
canonical and atomic term are well-formed with respect to a
well-formed signature, context and canonical type triple. 
%
Figure~\ref{fig:lf-judgements} presents the rules for deriving
these judgements.
%
In the rules \canonkindpi\ and \canontermlam\ we assume $x$ to be a
variable that does not appear free in $\Gamma$.
%
The formation rule for type and term level application,
i.e. $\atomfamapp$ and $\atomtermapp$, require the substitution of a
term into a kind or a type.
%
Use is made towards this end of hereditary substitution. The index for
such a substitution is obtained by erasure from the type established
for the term.

The judgement forms other than $\lfsig{\Sigma}$ that are described
above are parameterized by a signature that remains unchanged in the
course of their derivation.
%
In the rest of this paper we will assume a fixed signature that has in
fact been verified to be well-formed at the outset. 
%
The judgement forms require some of their other components to satisfy
additional restrictions. 
%
For example, judgements of the forms $\lfchecktype{\Gamma}{M}{A}$ and
$\lfsynthtype{\Gamma}{R}{A}$ require $\Sigma$, $\Gamma$ and $A$
to be well-formed as an ensemble.
%
To be coherent, the rules in Figure~\ref{fig:lf-judgements} must
ensure that in deriving a judgement that satisfies these requirements,
it is necessary only to consider the derivation of judgements that
also accord with these requirements.
%
The fact that they possess this property can be verified by an
inspection of their structure, using the observation that
hereditary substitution preserves the property of being well-formed
for kinds and types.  

% Arity typing judgements for terms approximate LF typing judgements as
% made precise below. 

% \begin{definition}
%   The arity context induced by the signature $\Sigma$  and context
%   $\Gamma$ is the collection of assignments that includes $x :
%   \erase{A}$ for each $x : A \in \Gamma$ and $c : \erase{A}$ for each
%   $c : A \in \Sigma$.
%   %
%   When the context $\Gamma$ is irrelevant or empty, we shall refer to
%   the arity context as the one induced by just $\Sigma$.
% \end{definition}

% \begin{theorem}\label{th:arityapprox}
%   Let $\STLCGamma$ be the arity context induced by the signature
%   $\Sigma$ and context $\Gamma$.
%   %
%   If $\lfctx{\Gamma}$ then $\Gamma$ respects $\Theta$.
%   %
%   If $\lfkind{\Gamma}{K}$ or $\lftype{\Gamma}{A}$ then, respectively,
%   $K$ or $A$ respect $\STLCGamma$.
%   %
%   If $\lfchecktype{\Gamma}{M}{A}$ is derivable, then
%   $\stlctyjudg{\STLCGamma}{M}{\erase{A}}$ must also be derivable.
%   %
%   If $\lfsynthtype{\Gamma}{R}{A}$ is derivable, then
%   $\stlctyjudg{\STLCGamma}{R}{\erase{A}}$ must also be derivable.
% \end{theorem}

% \begin{proof}
% The last two parts of the theorem are proved simultaneously by
% induction on the size of the derivation of
% $\lfchecktype{\Gamma}{M}{A}$ and $\lfsynthtype{\Gamma}{R}{A}$. 
% %
% The first two parts follows from them, again by induction on the
% derivation size.  
% \end{proof}

\subsection{Formalizing Object Systems in LF}\label{ssec:informal-reasoning}

\begin{figure*}[htbp]
\begin{tabular}{lcl}
  $\of{\tpty}{\type}$
  & \qquad & 
  $\of{\ofemptytm}{\ofty\app\emptytm\app\unittm}$

  \\
  
  $\of{\unittm}{\tpty}$
  & &

  \\

  $\of{\arrtm}{\arr{\tpty}{\tpty}}$
  & &
  $\of{\ofapptm}
      {\typedpi{E_1}{\tmty}{\typedpi{E_2}{\tmty}
      {\typedpi{T_1}{\tpty}{\typedpi{T_2}{\tpty}{}}}}}$
  \\
  & &
    \qquad
    $\typedpi{D_1}{\ofty\app E_1\app (\arrtm\app T_1\app T_2)}{
       \typedpi{D_2}{\ofty\app E_2\app T_1}{}}$

  \\

  $\of{\tmty}{\type}$
  & &
  \qquad $\ofty\app(\apptm\app E_1\app E_2)\app T_2$

  \\

  $\of{\emptytm}{\tmty}$
  & &
  \\

  $\of{\apptm}{\arr{\tmty}{\arr{\tmty}{\tmty}}}$ & &
    $\of{\oflamtm}
        {\typedpi{R}
        {\arr{\tmty}{\tmty}}
        {\typedpi{T_1}{\tpty}{\typedpi{T_2}{\tpty}{}}}}$
  \\

  $\of{\lamtm}{\arr{\tpty}{\arr{(\arr{\tmty}{\tmty})}{\tmty}}}$
  &  & 
  \qquad $\typedpi{D}{(\typedpi{x}{\tmty}
           {\typedpi{y}{\ofty\app x\app T_1}
           {\ofty\app(R\app x)\app T_2}})}{}$
  \\
  
  & & \qquad $\ofty\app(\lamtm\app T_1\app(\lflam{x}{R\app x}))
                              \app(\arrtm\app T_1\app T_2)$
  \\

  $\of{\ofty}{\arr{\tmty}{\arr{\tpty}{\type}}}$ & & \\  

  $\of{\eqty}{\arr{\tpty}{\arr{\tpty}{\type}}}$ & &
    $\of{\refltm}{\typedpi{T}{\tpty}{\eqty\app T\app T}}$
\end{tabular}
\caption{An LF Specification for the Simply-Typed Lambda Calculus}
\label{fig:stlc-term-spec}
\end{figure*}

A key use of LF is in formalizing systems that are described 
through relations between objects that are specified through a
collection of inference rules. 
%
In the paradigmatic approach, each such relation is represented by a
dependent type whose term arguments are encodings of objects that
might be in the relation in question.
%
The inference rules translate in this context into term constructors
for the type representing the relation.
%
We illustrate these ideas through an encoding of the typing relation
for the simply-typed $\lambda$-calculus (STLC), a running example for this
paper. 

We assume the reader to be familiar with the types and terms in the
STLC and also with the rules that define its typing relation.
%
Figure~\ref{fig:stlc-term-spec} presents an LF signature that serves as
an encoding of this system. 
%
This encoding uses the higher-order abstract syntax approach to
treating binding. 
%
The specification introduces two type families, $\tpty$ and $\tmty$ to 
represent the simple types and $\lambda$-terms.
%
Additionally, for each expression form in the object system, it
includes a constant that produces a term of type $\tpty$ or $\tmty$;
as should be apparent from the declarations, we have assumed an object
language whose terms are constructed from a single constant of
atomic type that is represented by the LF constant \emptytm\ and whose
type is represented by the LF constant \unittm.
%
This signature also provides a representation of two relations over
object language expressions: typing between terms and types and
equality between types. 
%
Specifically, the type-level constants $\ofty$ and $\eqty$ are
included towards this end. 
%
The rules defining the relations of interest in the object system are
encoded by constants in the signature.
%
The types associated with these constants ensure that well-formed
terms of atomic type that are formed using the constants correspond to
derivations of the relation in the object language that is represented
by the type. 

One of the purposes for constructing a specification is to use it to
prove properties about the object system.
%
For example, we may want to show that when a type can be associated
with a term in the STLC, it must be unique.
%
Based on our encoding, this property can be stated as the following
about typing derivations in LF:
\begin{quotation}
\noindent For any terms $M_1,M_2,E,T_1,T_2$, if there are LF
derivations for
$\lfchecktype{}{E}{\tmty}$, $\lfchecktype{}{T_1}{\tpty}$,
$\lfchecktype{}{T_2}{\tpty}$,
$\lfchecktype{}{M_1}{\ofty\app E\app T_1}$ and
$\lfchecktype{}{M_2}{\ofty\app E\app T_2}$, then there must be a term
$M_3$ such than there is a derivation for
$\lfchecktype{}{M_3}{\eqty\app T_1\app T_2}$. 
\end{quotation}
To prove this property, we would obviously need to unpack its logical
structure.
%
We would also need to utilize an understanding of LF in
analyzing the hypothesized typing derivations corresponding to the
STLC typing judgements. 
%
Considering the case where $E$ is an abstraction will lead us to actually
wanting to prove a more general property:
\begin{quotation}
\noindent For any terms $M_1,M_2,E,T_1,T_2$ and contexts $\Gamma$, if
there are LF derivations for the judgements
$\lfchecktype{\Gamma}{E}{\tmty}$, $\lfchecktype{\Gamma}{T_1}{\tpty}$,
$\lfchecktype{\Gamma}{T_2}{\tpty}$,
$\lfchecktype{\Gamma}{M_1}{\ofty\app E\app T_1}$ and
$\lfchecktype{\Gamma}{M_2}{\ofty\app E\app T_2}$, 
then there must be a term $M_3$ such than there is a derivation
for $\lfchecktype{}{M_3}{\eqty\app T_1\app T_2}$. 
\end{quotation}
Now, this property is not provable without some constraints on the form
of contexts.
%
In this example, it suffices to prove it when $\Gamma$ is restricted
to being of the form
\begin{center}
 $(x_1:\tmty,y_1:\ofty\app x_1\app{Ty}_1,\ldots,x_n:\tmty,y_n:\ofty\app x_n\app{Ty}_n)$.
\end{center}
%
\noindent In completing the argument, we would need to use properties of LF
derivability.
%
A property that would be essential in this case is the finiteness of
LF derivations, which enables us to use an inductive argument.

The objective in this paper is to provide a formal mechanism for
carrying out such analysis.
%
The first step in this direction would be the description of a logic
that permits the expression of the properties that we might want to
verify. 
%
We would then want to complement the logic with inference rules that 
permit the encoding of interesting and sound forms of reasoning. 

\subsection{A Synthetic Typing Rule}
%\subsection{Meta-Theoretic Properties about LF Derivability}
\label{ssec:lf-properties}

% Our reasoning system will need to embody an understanding of
% derivability in LF.  
% %
% We describe some properties related to this notion here that will be 
% useful in this context. 
% %
% The first three theorems, which express structural properties
% about derivations, have easy proofs.
% %
% The fourth theorem states a subsitutivity property for wellformedness
% judgements.
% %
% This theorem is proved in \cite{harper07jfp}.

% \begin{theorem}\label{th:weakening}
% If $\mathcal{D}$ is a derivation for $\lfkind{\Gamma}{K}$,
% $\lftype{\Gamma}{A}$ or $\lfchecktype{\Gamma}{M}{A}$, then, for any
% variable $x$ that is fresh to the judgement and for any $A'$ such that
% $\lftype{\Gamma}{A'}$ is derivable, there is a derivation,
% respectively, for $\lfkind{\Gamma, x:A'}{K}$,
% $\lftype{\Gamma, x : A'}{A}$ or $\lfchecktype{\Gamma, x: A'}{M}{A}$
% that has the same structure as $\mathcal{D}$. 
% \end{theorem}

% \begin{theorem}\label{th:strengthening}
%   If $\mathcal{D}$ is a derivation for judgements $\lfkind{\Gamma, x:A'}{K}$,
% $\lftype{\Gamma, x : A'}{A}$ or $\lfchecktype{\Gamma, x: A'}{M}{A}$
%   and $x$ is a variable that does not appear free in $K$, $A$, or $M$
%   and $A$ respectively, then there must be a derivation that has the
%   same structure as $\mathcal{D}$ for $\lfkind{\Gamma}{K}$,
%   $\lftype{\Gamma}{A}$ or $\lfchecktype{\Gamma}{M}{A}$, respectively.
% \end{theorem}

% \begin{theorem}\label{th:exchange}
% If $x$ does not appear in $A_2$ then
% $\Gamma_1,y:A_2,x:A_1,\Gamma_3$ is a well-formed context with respect
% to a signature $\Sigma$ whenever $\Gamma_1,x:A_1,y:A_2,\Gamma_3$ is. 
% %
% Further, if there is a derivation $\mathcal{D}$ for
% $\lfkind{\Gamma, x:A_1, y:A_2,\Gamma_2}{K}$,
% $\lftype{\Gamma, x:A_1, y:A_2,\Gamma_2}{A}$ or $\lfchecktype{\Gamma,
%   x:A_1, y:A_2,\Gamma_2}{M}{A}$, then there must be a derivation that
% has the same structure as $\mathcal{D}$ for
% $\lfkind{\Gamma, y:A_2, x:A_1,\Gamma_2}{K}$,
% $\lftype{\Gamma, y:A_2, x:A_1,\Gamma_2}{A}$ or $\lfchecktype{\Gamma,
%   y:A_2, x:A_1,\Gamma_2}{M}{A}$, respectively.
% \end{theorem}

% \begin{theorem}\label{th:transitivity}
% Assume that $\lfctx{\Gamma_1,x_0:A_0,\Gamma_2}$ and
% $\lfchecktype{\Gamma_1}{M_0}{A_0}$ have derivations, and let $\theta$ be
% the substitution $\{ \langle x_0, M_0,\erase{A_0} \rangle \}$. 
% %
% Then there is a $\Gamma'_2$ such that
% $\hsub{\theta}{\Gamma_2}{\Gamma'_2}$ and $\lfctx{\Gamma_1,\Gamma'_2}$
% have derivations. 
% %
% Further,
% \begin{enumerate}
%   \item if $\lfkind{\Gamma_1,x_0:A_0,\Gamma_2}{K}$ has a derivation,
%     then there is a $K'$ such that $\hsub{\theta}{K}{K'}$ and
%     $\lfkind{\Gamma_1,\Gamma'_2}{K'}$ have derivations;

%   \item if $\lftype{\Gamma_1,x_0:A_0,\Gamma_2}{A}$ has a derivation,
%     then there is an $A'$ such that $\hsub{\theta}{A}{A'}$ and
%     $\lftype{\Gamma_1,\Gamma'_2}{A'}$ have derivations; and

%   \item if $\lfchecktype{\Gamma_1,x_0:A_0,\Gamma_2}{M}{A}$ has a
%     derivation (for some well-formed type $A$), there is an $A'$ and
%     an $M'$ such that $\hsub{\theta}{A}{A'}$, $\hsub{\theta}{M}{M'}$,
%     and $\lfchecktype{\Gamma_1,\Gamma_2'}{M'}{A'}$ have derivations.
% \end{enumerate}
% \end{theorem}

A system for reasoning about LF derivability will need to build in the
ability to analyze judgements of the form $\lfchecktype{\Gamma}{M}{A}$.
%
Such an analysis can be driven by the structure of the type $A$.
%
The decomposition when $A$ is of the form $\typedpi{x_1}{A_1}{A_2}$
has an obvious form.
%
Theorem~\ref{th:atomictype}, which is proved
in~\cite{nadathur21arxiv}, provides the basis for the analysis when
$A$ is an atomic type.  

\begin{theorem}\label{th:atomictype}
Let $\Gamma$ be a context such that $\lfctx{\Gamma}$ has a derivation.
%
Then $\lfsynthtype{\Gamma}{R}{A'}$  has a derivation if and only if

\begin{enumerate}
\item $R$ is of the form $(c \app M_1 \app \ldots\app M_n)$ for some
    \[c:\typedpi{y_1}{A_1}{\ldots \typedpi{y_n}{A_n}{A}} \in \Sigma\] or
    of the form $(x \app M_1 \app \ldots\app M_n)$ for some
    \[x:\typedpi{y_1}{A_1}{\ldots \typedpi{y_n}{A_n}{A}} \in \Gamma,\]
    
  \item there is a sequence of types $A'_1,\ldots,A'_n$ such that, for
    $1\leq i\leq n$, there are derivations for
    \[\hsub{\{\langle y_1, M_1,\erase{A_1}\rangle, \ldots,
             \langle y_{i-1}, M_{i-1}, \erase{A_{i-1}} \rangle\}}
          {A_i}
          {A'_i}\]
    and $\lfchecktype{\Gamma}{M_i}{A'_i}$, and
         
  \item there are derivations for
    \[\hsub{\{\langle y_1, M_1,\erase{A_1}\rangle, \ldots,
                 \langle y_n, M_n, \erase{A_n} \rangle\}}
              {A}
              {A'}\]
    and $\lftype{\Gamma}{A'}$.
  \end{enumerate}
  Moreover, if $\lfsynthtype{\Gamma}{R}{A'}$ has a derivation of
  height $h$, then the derivations that are known to exist for
  $\lfchecktype{\Gamma}{M_i}{A'_i}$  for $1 \leq i \leq n$ by clause
  (2) must all have height less than $h$.
  \end{theorem}

Theorem~\ref{th:atomictype} gives us an alternative means for deriving
judgements of the form $\lfchecktype{\Gamma}{R}{P}$, in the process
dispensing with judgements of the form $\lfsynthtype{\Gamma}{R}{A}$.
%
Note also that in \emph{analyzing} judgements of the form
$\lfchecktype{\Gamma}{R}{P}$, it is necessary to consider only
\emph{shorter} derivations for subterms of $R$.
%
This observation provides a means for arguing inductively on the
heights of LF derivations.  

