\section{A Logic for Expressing Properties of LF Specifications}
\label{sec:logic}

The discussions in Section~\ref{ssec:informal-reasoning} suggest a 
possible structure for a logic for stating properties of an object
system that has been described using LF. 
%
The logic would be parameterized by a well-formed LF signature.
%
Typing judgements interpreted via LF derivability would serve as 
atomic formulas.
%
More complex formulas would be constructed using 
logical connectives and quantifiers over LF terms.
%
Quantification would need to be allowed not only over term variables
but also over variables representing LF contexts. 

To develop an actual logic based on these ideas, we must describe a
more precise correspondence between LF typing judgements and atomic
formulas. 
%
The judgement forms that must considered in this context 
are $\lfchecktype{\Gamma}{M}{A}$ and $\lfsynthtype{\Gamma}{R}{A}$.
%
The main judgement form is in fact the first one and, as we have just
noted, Theorem~\ref{th:atomictype} allows us to identify a special
``focused'' typing rule that obviates the second judgement form. 
%
Judgement of the first kind in LF assume the wellformedness of the
context $\Gamma$ and the type $A$.
%
In the logic, the context and, therefore, also the type can be
dynamically determined by instantiations for context variables.
%
To deal with this situation, we will build the wellformedness of
$\Gamma$ and $A$ into the interpretation of the encoding of the
judgement. 
%
There is, however, an aspect of the wellformedness checking that we
would like to extract into a static pre-processing phase.
%
The LF typing rules combine the checking of canonicity of terms with
the determination of inhabitation that relies on the semantically more
meaningful aspect of dependencies in types.
%
To allow the focus in the logic to be on the latter aspect, we will
build the former into a wellformedness criterion for formulas using
arity types.

Another aspect that needs further consideration is the representation of
LF contexts in atomic formulas.
%
To support typing derivations that use the
\canontermlam\ rule, this representation must allow for the explicit
association of types with variables.
%
These variables may appear free in the terms and types in the atomic
formula.
%
However, their interpretation in this context must be different from
the variables that are bound by quantifiers: in particular, these
variables cannot be instantiated and each of them must be treated as
being distinct within the atomic formula. 
%
The necessary treatment of these variables can be realized by
representing them by \emph{nominal constants} in the style of
\cite{gacek11ic,tiu06lfmtp}.
%
Context expressions must, in addition, allow for an unspecified part
whose exact extent is to be determined by instantiation of an external
context quantifier.
%
To support this ability, we will allow context variables to appear in
these expressions.
%
However, as observed in Section~\ref{ssec:informal-reasoning}, we
would like to be able to restrict the instantiation of such variables
to blocks of declarations adhering to specified forms.
%
To impose such constraints, the logic will permit context variables to
be typed by \emph{context schemas} that are motivated by regular world
descriptions used in the Twelf
system~\cite{Pfenning02guide,schurmann00phd}. 

In the rest of this section, we present the logic \logic\ that
substantiates the ideas outlined above.
%
The first two subsections present the well-formed formulas of \logic and
identify their intended meaning. 
%
The end result of this discussion is a means for describing properties
of a specification given by an LF signature and for assessing the
validity of such properties.
%
The third subsection illuminates this capability through a
collection of examples.
%% %
%% The last subsection observes the counterpart in \logic\ of the
%% property of irrelevance of the particular names that are chosen for
%% the variables bound by the context in an LF judgement.
%% %
%% The particular expression of this property takes the form of the
%% invariance of validity of formulas under permutations of
%% nominal constants. 

\subsection{The Formulas of the Logic}

Figure~\ref{fig:logic-terms-and-types} presents the syntax of the
expressions in \logic\ that represent LF types and terms.
%
As with LF syntax, we use $c$ and $d$ to represent term level
constants, $a$ and $b$ to represent type level constants and $x$ and
$y$ to represent term-level variables.
%
We also use $n$ to represent a special category of symbols called the
nominal constants.
%
LF terms and types are obviously a subset of the expressions presented
here.
%
Going the other way, there are two main additions to the LF
counterparts in the collection of expressions described here.
%
First, nominal constants may be used in constructing terms.
%
Second, as we shall soon see, variables may be bound not only by
term and type level abstractions but also by formula level
quantifiers.

\begin{figure}[tbhp]
\[
\begin{array}{r r c l}
  \mbox{\bf Terms} & M,N & ::= & R\ |\ \lflam{x}{M}\\
  \mbox{\bf Atomic Terms} & R & ::= & c\ |\ x\ |\ n\ |\ R\app M\\[5pt]
  \mbox{\bf Types} & A & ::= &
           P\ |\ \typedpi{x}{A_1}{A_2}\\
  \mbox{\bf Atomic Types} & P & ::= & a\ |\ P\app M\\[5pt]
\end{array}
\]
\caption{Terms and Types in \logic}
\label{fig:logic-terms-and-types}
\end{figure}

The logic \logic\ is parameterized by a signature $\Sigma$
that assigns kinds to type-level constants and types to term-level
ones and that is well-formed in the sense described in
Section~\ref{sec:lf}.
%
The logic also assumes as given a set $\noms$ of nominal constants,
each specified with an arity type, with a countably infinite supply of
such constants for each arity type $\alpha$.
%
Since the types of the nominal constants are fixed once and for all by
$\noms$, we may treat sets of such constants ambiguously as
collections whose elements are of the form $n : \alpha$ or simply
$n$. 
%

\begin{figure}[tbhp]

\begin{center}
\begin{tabular}{c}

\infer{\akindingp{\STLCGamma}{a}{K}}
      {a:K \in \Sigma}

%\qquad\qquad
\\[10pt]      

\infer{\akindingp{\STLCGamma}{P\app M}{K}}
      {\akindingp{\STLCGamma}{P}{\typedpi{x}{A}{K}} \qquad
       \stlctyjudg{\STLCGamma}{M}{\erase{A}}}

\\[10pt]

\infer{\wftype{\STLCGamma}{P}}
      {\akindingp{\STLCGamma}{P}{\type}}

%\qquad\qquad
\\[10pt]      

\infer{\wftype{\STLCGamma}{\typedpi{x}{A_1}{A_2}}} 
      {\wftype{\STLCGamma}{A_1} \qquad \wftype{\aritysum{\{x :
            \erase{A_1} \}}{\STLCGamma}}{A_2}}
\end{tabular}
\end{center}

\caption{Arity Kinding for Canonical Types}
\label{fig:arity-kinding}
\end{figure}

Expressions in \logic\ are expected to
satisfy typing constraints that check for canonicity.
%
At the term level, these constraints will be realized through arity
typing relative to a suitable arity context.
%
At the type level, we must additionally ensure that (type) constants
have been supplied with an adequate number of arguments.
%
We make these notions precise below; we assume the obvious extension
of erasure to types in \logic\ here and elsewhere.
%
\begin{definition}
The typing relation between an arity context, a term and an arity type
that is described in Definition~\ref{def:aritytyping} is extended to
the present context by permitting terms to contain nominal constants
and by allowing arity contexts to contain assignments to such
constants.
%
The rules in Figure~\ref{fig:arity-kinding} define an arity kinding
property denoted by $\wftype{\STLCGamma}{A}$ for a type $A$ relative to
an arity context $\STLCGamma$.
%
In these rules, $\Sigma$ is the signature parameterizing \logic.
%
We will often need to refer to the arity context induced by $\Sigma$.
%
We call this the \emph{initial constant context} and we reserve the symbol
$\STLCGamma_0$ to denote it.
\end{definition}

Hereditary substitution extends naturally to the terms and types in
\logic\ by treating nominal constants like other constants.
%
The following theorem relating to such substitutions has an obvious
proof.  

\begin{theorem}\label{th:aritysubs-ty}
If $\theta$ is type preserving with respect to $\STLCGamma$ and
$\wftype{\aritysum{\context{\theta}}{\STLCGamma}}{A}$ and
$\hsub{\theta}{A}{A'}$ have derivations, then
$\wftype{\STLCGamma}{A'}$ has a derivation.
\end{theorem}

\begin{figure}[tbhp]
  \[\begin{array}{rrcl}
\mbox{\bf Block Declarations} & \Delta & ::= & \emptybb\ \vert\ \Delta, y : A \\
\mbox{\bf Block Schema}   & \mathcal{B} & ::= & \{x_1:\alpha_1,\ldots, x_n:\alpha_n\}\Delta\\
\mbox{\bf Context Schema} & \mathcal{C} & ::= & \emptycs\ \vert\ \mathcal{C}, \mathcal{B}
\end{array}\]
\caption{Block Schemas and Context Schemas}
\label{fig:context-schemas}
\end{figure}

The logic allows for quantifiers over LF contexts.
%
In the intended interpretation, such quantifiers are meant to be
instantiated with context expressions that assign LF types to nominal
constants. 
%
However, we will want to constrain such instantiations.
%
This ability is supported by typing
context quantifiers using \emph{context schemas} whose structure is
presented in Figure~\ref{fig:context-schemas}.
%
In essence, a context schema comprises a collection of \emph{block
  schemas}.
%
A block schema consists of a header of variables annotated with
arity types and a body of declarations associating types with
variables. 
%
Each variable in the header and that is assigned a type in the body of
a block schema is required to be distinct. 
%
A block is intended to serve as a template for generating a sequence
of bindings for nominal constants through an instantiation process
that will be made clear in the next subsection.
%
A context expression corresponding to a context schema is to be obtained
by some number of instantiations of its block schemas.
%
Block and context schemas are required to satisfy typing constraints
towards ensuring that the context expressions generated from them will
be well-formed in the manner required by the logic.
%
These constraints are represented by the typing judgements
$\abstyping{\mathcal{B}}$ and $\acstyping{\mathcal{C}}$, respectively,
that are defined by the rules in Figure~\ref{fig:schematyping}.

\begin{figure*}[tbhp]

\begin{center}
\begin{tabular}{c}

\infer{\wfdecls{\STLCGamma}{\emptybb}{\STLCGamma}}{}

\qquad\qquad

\infer{\wfdecls{\STLCGamma}{\Delta, y:A}{\STLCGamma' \cup \{y:\erase{A}\}}} 
      {\wfdecls{\STLCGamma}{\Delta}{\STLCGamma'} \qquad 
       y\ \mbox{\rm is not assigned by}\ \STLCGamma' \qquad
       \wftype{\STLCGamma'}{A}}

\\[15pt]

\infer{\abstyping{\{x_1:\alpha_1,\ldots, x_n:\alpha_n\}\Delta}}
      {x_1,\ldots,x_n\ \mbox{\rm are distinct variables}
       \qquad
       \wfdecls{\STLCGamma_0 \cup \{x_1 : \alpha_1, \ldots,
                                    x_n : \alpha_n\}}
               {\Delta}
               {\STLCGamma'}}

\\[15pt]

\infer{\acstyping{\emptycs}}{}

\qquad

\infer{\acstyping{\mathcal{C},\mathcal{B}}}
      {\acstyping{\mathcal{C}} \qquad \abstyping{\mathcal{B}}}

\end{tabular}
\end{center}
\caption{Wellformedness Judgements for Block and Context Schemas}
\label{fig:schematyping}
\end{figure*}


\begin{figure}[tbhp]
\[\begin{array}{lrcl}
\mbox{\bf Context Expressions} & G & ::= &
    \emptyce\ |\ \Gamma\ |\ G,n:A\\
\mbox{\bf Formulas} & F & ::= & \fatm{G}{\of{M}{A}}\ |\ \ftrue\ |\ \ffalse\ |\\
& & & \myfor{F_1}{F_2}\ |\ \fand{F_1}{F_2}\ |\ \fimp{F_1}{F_2}\ |\\
& & &\fctx{\Gamma}{\mathcal{C}}{F}\ |\ \fall{x:\alpha}{F}\ |\ \fexists{x:\alpha}{F}
\end{array}\]
\caption{The Formulas of \logic}
\label{fig:formula-syntax}
\end{figure}

The syntax of the formulas of \logic\ is presented in
Figure~\ref{fig:formula-syntax}.
%
The symbol $\Gamma$ is used in these formulas to represent context
variables.
%
Atomic formulas, which represent LF typing judgements, have the form
$\fatm{G}{M:A}$.
%
The context in these formulas is constituted by a sequence of type
associations with nominal constants, possibly preceded by a context
variable.
%
Included in the collection are the logical constants $\ftrue$ and
$\ffalse$ and the familiar connectives for constructing more complex
formulas.
%
Universal and existential quantification over term variables is also
permitted and these are written as $\fall{x:\alpha}{F}$ and
$\fexists{x:\alpha}{F}$, respectively.
%
Such quantification is indexed, as might be expected, by arity types.
%
The collection also includes universal quantification over context
variables that is typed by context schemas, written as
$\fctx{\Gamma}{\mathcal{C}}{F}$.
%
We assume the usual principle of equivalence under renaming with
respect to the term and context quantifiers and apply them as needed. 

\begin{figure*}[tbhp]

\begin{center}
\begin{tabular}{c}

\infer{\wfctx{\STLCGamma}{\Xi}{\emptyce}}
      {} 
\qquad

\infer{\wfctx{\STLCGamma}{\Xi}{\Gamma}}
      {\Gamma \in\Xi}

\qquad

\infer{\wfctx{\STLCGamma}{\Xi}{G,n:A}}
      {\wfctx{\STLCGamma}{\Xi}{G} \qquad
       n:\erase{A}\in \STLCGamma \qquad
       \wftype{\STLCGamma}{A}}
\\[10pt]

\infer{\wfform{\STLCGamma}{\Xi}{\fatm{G}{M:A}}}
      {\wfctx{\STLCGamma}{\Xi}{G} \qquad
       \wftype{\STLCGamma}{A} \qquad
       \stlctyjudg{\STLCGamma}{M}{\erase{A}}}

\qquad

\infer{\wfform{\STLCGamma}{\Xi}{\ftrue}}{} 

\qquad
      
\infer{\wfform{\STLCGamma}{\Xi}{\ffalse}}{}

\\[10pt]
      
\infer[\bullet \in \{\supset,\land,\lor\}]
      {\wfform{\STLCGamma}{\Xi}{F_1 \bullet F_2}}
      {\wfform{\STLCGamma}{\Xi}{F_1} \qquad 
       \wfform{\STLCGamma}{\Xi}{F_2}}

\qquad

\infer{\wfform{\STLCGamma}{\Xi}{\fctx{\Gamma}{\mathcal{C}}{F}}}
      {\acstyping{\mathcal{C}} \qquad
       \wfform{\STLCGamma}{\Xi \cup \{ \Gamma \}}{F}}

\qquad
      
\infer[\genericq \in \{\forall, \exists \}]
      {\wfform{\STLCGamma}{\Xi}{\fgeneric{x:\alpha}{F}}}
      {\wfform{\aritysum{\{x:\alpha\}}{\STLCGamma}}{\Xi}{F}}
 
\end{tabular}
\end{center}
 
\caption{The Wellformedness Judgement for Formulas}
\label{fig:wfform}
\end{figure*}

A formula $F$ is determined to be well-formed or not relative to an arity
context $\STLCGamma$ and a collection of context variables $\Xi$.
%
This judgement is written concretely as $\wfform{\STLCGamma}{\Xi}{F}$
and the rules defining it are presented in Figure~\ref{fig:wfform}.
%
At the top-level, formulas are expected to be \emph{closed}, \ie, to not have
any free term or context variables.
%
More specifically, we expect $\wfform{\noms \cup \STLCGamma_0}{\emptyset}{F}$ to be
derivable for such formulas. 
%
The analysis within the scope of term and context
quantifiers augments these sets in the expected way.
%
For context quantifiers, this analysis must also check that the
annotating context schema is well-formed.
%
An atomic formula $\fatm{G}{M:A}$ is deemed well-formed if its
components $G$, $M$ and $A$ are well-formed and if $M$ can be assigned
the erased form of $A$ as its arity type.
%
The context expression $G$ is well-formed if any context variable used
in it is bound in the overall formula and if the types assigned to
nominal constants in the explicit part of $G$ are well-formed and
such that their erased forms match the arity types of the nominal
constants they are assigned to.
%
Note that these types may use nominal constants without paying
attention to dependency ordering; assessing
whether they are used in a manner that respects this ordering is a part
of the meaning of the atomic formula.

%% The following theorem, whose proof is obvious, shows that
%% the wellformedness judgement for formulas continues to
%% hold under the augmentation of the two contexts that  parameterize it. 
%% %
%% \begin{theorem}\label{th:wfsupset}
%% If $\wfform{\STLCGamma}{\Xi}{F}$ has a derivation and $\STLCGamma
%% \subseteq \STLCGamma'$ and $\Xi \subseteq \Xi'$, then
%% $\wfform{\STLCGamma'}{\Xi'}{F}$ also has a derivation.
%% \end{theorem}

\subsection{The Interpretation of Formulas}

A key component to understanding the meanings of formulas is
understanding the interpretation of the quantifiers over term and
context variables.
%
These quantifiers are intended to range over closed expressions of the
relevant categories.
%
For a quantifier over a term variable, this translates concretely into
closed terms of the relevant arity type.
%
For a quantifier over a context variable, we must first explain when
a context expression satisfies a context schema. 

\begin{figure*}[tbhp]

\begin{center}
\begin{tabular}{c}

\infer{\declinst{\mathbb{N}}{\emptybb}{\emptyce}{\emptyset}}{}

\qquad

\infer{\declinst{\mathbb{N}}{\Delta,y:A}{G, n : A'}
                 {\theta \cup \{\langle y,n,\erase{A}\rangle\}}}
      {\declinst{\mathbb{N}}{\Delta}{G}{\theta} \qquad
        n : \erase{A} \in {\mathbb{N}} \qquad
        \hsub{\theta}{A}{A'}}

\\[15pt]

\infer{\bsinst{\mathbb{N}}{\Psi}{\{x_1 : \alpha_1,\ldots, x_n : \alpha_n\}\Delta}{G}}
      {\declinst{\mathbb{N}}{\Delta}{G'}{\theta}
       \quad\ 
       \{ \stlctyjudg{{\mathbb{N}} \cup \Psi \cup \STLCGamma_0}{t_i}{\alpha_i}\ \vert\ 1 \leq i \leq n \}
       \quad\ 
       \hsub{\{\langle x_i,t_i,\alpha_i\rangle \ \vert\ 1 \leq i \leq n \}} 
             {G'}
             {G}
       }


\\[15pt]

\infer{\csinstone{\mathbb{N}}{\Psi}{\mathcal{C},\mathcal{B}}{G}}
      {\bsinst{\mathbb{N}}{\Psi}{\mathcal{B}}{G}}

\qquad

\infer{\csinstone{\mathbb{N}}{\Psi}{\mathcal{C},\mathcal{B}}{G}}
      {\csinstone{\mathbb{N}}{\Psi}{\mathcal{C}}{G}}

\qquad

\infer{\csinst{\mathbb{N}}{\Psi}{\mathcal{C}}{\emptyce}}
      {}

\qquad 

\infer{\csinst{\mathbb{N}}{\Psi}{\mathcal{C}}{G, G'}}
      {\csinst{\mathbb{N}}{\Psi}{\mathcal{C}}{G} \qquad
       \csinstone{\mathbb{N}}{\Psi}{\mathcal{C}}{G'}}

\end{tabular}
\end{center}

\caption{Instantiating a Context Schema}
\label{fig:ctx-schema}
\end{figure*}

We do this by describing the relation of ``being an instance of''
between a closed context expression $G$ and a context schema $\mathcal{C}$.
%
This relation is indexed by a nominal constant context $\mathbb{N}$
that is a subset of $\noms$ and a \emph{term variables context} $\Psi$
that identifies a finite collection of such variables together with their
arity types: in combination with  the constants in $\STLCGamma_0$,
these collections, circumscribe the symbols that can be used in the
declarations in the context expressions.\footnote{In 
  the use of this relation in this paper, $\mathbb{N}$ will
  be $\noms$ and $\Psi$ will be the empty set. The more general form
  for this relation, which includes a parameterization by these sets,
  is needed in the development of the proof system
  in~\cite{nadathur21arxiv}. We retain that form for consistency.} 
The relation is written as $\csinst{\mathbb{N}}{\Psi}{\mathcal{C}}{G}$
and it is defined by the rules in Figure~\ref{fig:ctx-schema}.
%
This relation is defined via the repeated use of a ``one-step''
instantiation relation written as
$\csinstone{\mathbb{N}}{\Psi}{\mathcal{C}}{G}$; note that by $G, G'$
we mean a context expression that is obtained by adding the bindings
corresponding to $G'$ in front of those in $G$.
%
The definition of the one-step instantiation relation for context
schemas uses an auxiliary judgement
$\bsinst{\mathbb{N}}{\Psi}{\mathcal{B}}{G}$ that denotes the relation
of ``being an instance of'' between a block 
schema and a context expression fragment.
%
This relation holds when the context expression is obtained by
generating a sequence of bindings for nominal constants from
$\mathbb{N}$ using the body of the block schema and then instantiating
the variables in the header of the block schema with terms of the
right arity types.
%
The former task is realized through the relation
$\declinst{\mathbb{N}}{\Delta}{G}{\theta}$ that holds between a block of
declarations $\Delta$, a context expression $G$ that is obtained
by replacing the variables assigned in $\Delta$ with
suitable nominal constants, and a substitution $\theta$ that
corresponds to this replacement.
%
We assume here and elsewhere that the application of a hereditary
substitution to a sequence of declarations  corresponds to its
application to the type in each assignment. 

The following theorem, which states a coherence property for the
instantiation of a context schema, is proved
in~\cite{nadathur21arxiv}.

\begin{theorem}\label{th:schemainst}
Let $\mathcal{C}$ and $G$ be a context schema and a context expression
such that $\acstyping{\mathcal{C}}$ and
$\csinst{\mathbb{N}}{\Psi}{\mathcal{C}}{G}$ are derivable. Then for
any arity context $\STLCGamma$ such that
$\mathbb{N} \cup \Psi \cup \STLCGamma_0 \subseteq \STLCGamma$, it is the case that 
$\wfctx{\STLCGamma}{\emptyset}{G}$ has a derivation. 
\end{theorem}

%% \begin{proof}
%% We first show that for any block declaration
%% $\Delta$ and any arity context $\STLCGamma$ such that $\mathbb{N}
%% \subseteq \STLCGamma$, if $\wfdecls{\STLCGamma}{\Delta}{\STLCGamma'}$ and
%% $\declinst{\mathbb{N}}{\Delta}{G'}{\theta'}$ are derivable for some $\STLCGamma'$
%% and $\theta'$, then (a)~$\theta'$ is type preserving with respect to
%% $\Theta$, (b)~$\Theta'$ is $\aritysum{\context{\theta'}}{\Theta}$, and
%% (c)~each binding in $G'$ is of the form $n : A$ where $n:\erase{A}\in
%% \Theta$ and $\wftype{\Theta}{A}$ has a derivation.
%% %
%% This claim is proved by induction on the derivation of
%% $\wfdecls{\STLCGamma}{\Delta}{\STLCGamma'}$; properties (a) and (b)
%% are included in the claim because they are useful together with
%% Theorem~\ref{th:aritysubs-ty} in showing property (c) in the
%% induction step. 
%% %
%% Next we show, through an easy inductive argument, that if
%% $\wfdecls{\STLCGamma_0 \cup \{x_1 : \alpha_1, \ldots,
%%                               x_n : \alpha_n\}}
%%          {\Delta}
%%          {\STLCGamma'}$
%% has a derivation and $\STLCGamma$ is such that
%% $\mathbb{N} \cup \Psi \cup \STLCGamma_0 \subseteq \STLCGamma$, then, for
%% some $\STLCGamma''$, it is the case that 
%% $\wfdecls{\aritysum{\{x_1 : \alpha_1, \ldots,x_n : \alpha_n\}}
%%                    {\STLCGamma}}           
%%          {\Delta}
%%          {\STLCGamma''}$
%% has a derivation.
%% %
%% Using Theorem~\ref{th:aritysubs-ty} with these two
%% observations, we can show easily that if
%% $\abstyping{\{x_1 : \alpha_1,\ldots, x_n : \alpha_n\}\Delta}$ and 
%% $\bsinst{\mathbb{N}}{\Psi} 
%%         {\{x_1 : \alpha_1,\ldots, x_n : \alpha_n\}\Delta}
%%         {G}$ 
%% have derivations then for each binding of the form $n:A$ in $G$ it is
%% the case that $n:\erase{A} \in \Theta$ and $\wftype{\Theta}{A}$.
%% %
%% The theorem follows easily from this observation.
%% \end{proof}

In defining validity for formulas, we will need to consider
substitutions for context and term variables.
%
Context variables substitutions have the form
$\{G_1/\Gamma_1,\ldots,G_n/\Gamma_n\}$ where, for $1 \leq i \leq n$,
$\Gamma_i$ is a context variable and $G_i$ is a context expression. 
%
If $\sigma$ is such a substitution, we will write $\domain{\sigma}$ to
denote the set $\{\Gamma_1,\ldots,\Gamma_n\}$.
%
Further, the application of $\sigma$ to a formula
$F$, which is denoted by $\subst{\sigma}{F}$, will correspond to the 
replacement of the free occurrences of the variables
$\Gamma_1,\ldots,\Gamma_n$ in $F$ by the corresponding context
expressions, renaming bound context variables appearing in $F$ away
from those appearing in $G_1,\ldots,G_n$.
%
For term variables, the replacement must also ensure the
transformation of the resulting expression to normal form.
%
Towards this end, we adapt hereditary substitution to formulas.
%
The application of this substitution simply distributes over
quantifiers and logical symbols, respecting the scopes of quantifiers
through the necessary renaming.
%
The application to the atomic formula $\fatm{G}{M:A}$ also distributes
to the component parts.
%
We have already discussed the application to terms and types.
%
The application to context expressions leaves context variables
unaffected and simply distributes to the types in the explicit
bindings.
%
Note that no check is mandated in the process for clashes in the names
of nominal constants appearing in the context expression being
substituted into and the substitution terms. 
%
%% In this respect, this application is unlike that to LF contexts
%% that is defined in Figure~\ref{fig:hsubctx}.


\begin{theorem}\label{th:subst-formula}
Let $\STLCGamma$ be an arity context and let $\Xi$ be a collection of
context variables.
\begin{enumerate}
\item If $\theta$ is a term variables substitution that is arity type preserving with
respect to $\STLCGamma$ and $F$ is a formula such that there is a
derivation for $\wfform{\aritysum{\context{\theta}}{\STLCGamma}}{\Xi}{F}$,
then there is a unique formula $F'$ such that
$\hsub{\theta}{F}{F'}$ has a derivation.
%
Moreover, for this $F'$ it is the case that
$\wfform{\STLCGamma}{\Xi}{F'}$ is derivable.  

\item If $\sigma=\{G_1/\Gamma_1,\ldots,G_n/\Gamma_n\}$ is a context variables
substitution which is such that all judgements in the collection
\[\left\{\wfctx{\STLCGamma}
             {\Xi\setminus\{\Gamma_1,\ldots,\Gamma_n\}}{G_i}\ |\ 
                    1\leq i\leq n\right\}\]
are derivable and $F$ is a formula such that there is a derivation for
$\wfform{\STLCGamma}{\Xi}{F}$, then there is a derivation for
\[\wfform{\STLCGamma}{\Xi\setminus\{\Gamma_1,\ldots,\Gamma_n\}}{\subst{\sigma}{F}}.\]
\end{enumerate}
\end{theorem}

\begin{proof}
The first clause follows from an induction on the derivation of 
$\wfform{\aritysum{\context{\theta}}{\STLCGamma}}{\Xi}{F}$, using  
Theorems~\ref{th:uniqueness} and \ref{th:aritysubs} in the atomic case
to ensure the appropriate arity typing judgements will be derivable under 
the substitution $\theta$.
%
The second clause follows from an induction on the derivation of
$\wfform{\STLCGamma}{\Xi}{F}$, using the assumption that
$\wfctx{\STLCGamma}{\Xi\setminus\{\Gamma_1,\ldots,\Gamma_n\}}{G_i}$
is derivable to ensure wellformedness under the substitution $\sigma$
in the atomic case.
\end{proof}

\noindent Following the notation introduced after Theorem~\ref{th:aritysubs}, 
if $F$ and $\theta$ are a formula and a substitution that together satisfy
the requirements of the first part of the theorem, we will write
$\hsubst{\theta}{F}$ to denote the $F'$ for which 
$\hsub{\theta}{F}{F'}$ is derivable.
%
%% As is implicit in the preceding discussion, term and context variables
%% substitutions may introduce new nominal constants. 
%% %
%% If $\theta$ is a term variables substitution, we will write
%% $\supportof{\theta}$ to denote the collection of such constants that
%% appear in the terms in $\range{\theta}$.
%% %
%% Similarly, if $\sigma$ is the context variables substitution
%% $\{G_1/\Gamma_1,\ldots,G_n/\Gamma_n\}$, 
%% we will write $\supportof{\sigma}$ to denote the collection of nominal
%% constants that appear in $G_1,\ldots,G_n$.

A closed atomic formula of the form
$\fatm{G}{M:A}$ is intended to encode an LF judgement of the
form $\lfchecktype{\Gamma}{M}{A}$.
%
In this encoding, nominal constants that appear in terms represent
free variables for which bindings appear in the context in LF
judgements.
%
To support this interpretation, the LF typing rules that introduce
variables into context must use nominal constants instead.
%
We build in such a modification to
define validity for closed atomic formulas with one further
qualification: unlike in the LF judgement, 
for the atomic formula we must also ascertain the wellformedness of
the context and the type.
%
This notion of validity is then extended to all closed formulas by
recursion on formula structure.

\begin{definition}\label{def:semantics}
Let $F$ be a formula such that $\wfform{\noms \cup
  \STLCGamma_0}{\emptyset}{F}$ is derivable. 
%
\begin{itemize}
\item If $F$ is $\fatm{G}{M:A}$, then it is valid exactly when $\lfctx{G}$,
  $\lftype{G}{A}$, and $\lfchecktype{G}{M}{A}$ are derivable in LF,
  under the interpretation of nominal constants as variables bound in
  a context and with the modification of the rules \canonkindpi, \canonfampi\ and
  \canontermlam\ to introduce fresh nominal constants into contexts and to
  instantiate the relevant bound variables in kinds, types and terms with
  these constants.

\item If $F$ is $\ftrue$ it is valid and if it is $\ffalse$ it is not valid.

\item If $F$ is $\fimp{F_1}{F_2}$, it is valid if $F_2$ is valid in
  the case that $F_1$ is valid.

\item If $F$ is $\fand{F_1}{F_2}$, it is valid if both $F_1$ and
  $F_2$ are valid.

\item If $F$ is $\myfor{F_1}{F_2}$, it is valid if either $F_1$ or $F_2$ is valid.

\item If $F$ is $\fctx{\Gamma}{\mathcal{C}}{F}$, it is valid if
  $\subst{\{G/\Gamma\}}{F}$ is valid for every $G$ such that
  $\csinst{\noms}{\emptyset}{\mathcal{C}}{G}$ is derivable.

\item If $F$ is $\fall{x:\alpha}{F}$, it is valid if
  $\hsubst{\{\langle x, M,\alpha\rangle \}}{F}$ is valid for every $M$ such that
  $\stlctyjudg{\noms \cup \STLCGamma_0}{M}{\alpha}$ is derivable.

\item If $F$ is $\fexists{x:\alpha}{F}$, it is valid if
  $\hsubst{\{\langle x, M,\alpha\rangle \}}{F}$ is valid for some $M$ such that 
  $\stlctyjudg{\noms \cup \STLCGamma_0}{M}{\alpha}$ is derivable.
\end{itemize}
\end{definition}

Theorems~\ref{th:schemainst} and \ref{th:subst-formula} ensure the
coherence of this definition. 

\subsection{Understanding the Notion of Validity}\label{ssec:logic-examples}

In the examples we consider below, we assume an instantiation of
\logic\ 
based on the signature presented in
Section~\ref{ssec:informal-reasoning}.
%
Obviously, any LF typing judgement based on that signature is expressable
in the logic.
%
Moreover, the corresponding formula will be valid exactly when the
typing judgement is derivable in LF.
%
Thus, the formulas
$\fatm{\emptyce}{\emptytm : \tmty}$,
$\fatm{\emptyce}{(\lamtm\app \unittm\app (\lflam{x}{x})) : \tmty}$ and
$\fatm{n : \tmty}{n:\tmty}$ are all valid. 
  %
Similarly, the formulas
$\fexists{d:\oty}{\fatm{\emptyce}{d : (\ofty \app \emptytm\app
    \unittm)}}$ and
\begin{tabbing}
\qquad\=\kill
\>$\fexists{d:\oty}{\fatm{\emptyce}{d :\ofty \app (\lamtm \app \unittm\app
    (\lflam{x}{x}))\app (\arrtm\app \unittm\app \unittm)}}$
\end{tabbing}
are valid but the formula
$\fexists{d:\oty}{\fatm{\emptyce}{d:\ofty \app
    (\lamtm \app \unittm\app (\lflam{x}{x}))\app \unittm}}$
is not.
%
Note that the arity type associated with the quantified variable in
each of these formulas provides only a rough constraint on the
instantiation needed to verify the validity of the formula; to do
this, the instance must also satisfy LF typeability requirements
represented by formula that appears within the scope of the
quantifier.

Wellformedness conditions for formulas ensure only that the terms
appearing within formulas satisfy canonicity requirements, i.e. that
these terms are in $\beta$-normal form and that variables and
constants are applied to as many arguments as they can take.
%
Arity typing does not distinguish between terms in different
expression categories.
%
For example, the formula
\begin{tabbing}
\qquad\=\kill
\> $\fexists{d:\oty}{\fatm{\emptyce}{d:\ofty \app (\lamtm \app \emptytm \app
    (\lflam{x}{x}))\app (\arrtm\app \unittm\app \unittm)}}$
\end{tabbing}
is well-formed but not valid.
%
An alternative design choice, with equivalent consequences from the
perspective of the valid properties that can be expressed in the
logic, might have been to let the fact that $\lamtm$ is ill-applied to
$\emptytm$ to impact on the wellformedness of the formula.
%
The wellformedness conditions do not also enforce a distinctness
requirement for bindings in a context.
%
Thus, the formula $\fatm{n : \tmty, n : \tpty}{\emptytm : \tmty}$ is
well-formed.
%
However, it is not valid because $\lfctx{n : \tmty, n : \tpty}$ is not
derivable in LF under the described interpretation for nominal
constants.
%
An implication of these observations is that a naive form of weakening 
does not hold with respect to the encoding of LF derivability in
\logic; additional conditions similar to this described in 
Theorem~\ref{th:weakening} must be verified for this principle to
apply.

To provide a more substantive example of the kinds of properties that
can be expressed in \logic, let us consider the formal statement of
the property of uniqueness of type assignments for the STLC.
%
As noted in Section~\ref{ssec:informal-reasoning}, this property is
best described in a form that considers typing expressions in
contexts that have a particular kind of structure. 
%
That structure can be formalized in \logic\ by a context
schema comprising the single block
\begin{tabbing}
  \qquad\=\kill
  \> $\{t : o\}x:tm,y:\ofty\app x\app t$.
\end{tabbing}
%
Let us denote this context schema by $c$.
%
Observe that a context that instantiates this schema will not
provide a variable that can be used to construct an atomic term of
type $\tpty$.
%
Thus, the strengthening property for expressions representing types
that is expressed by the formula 
\begin{tabbing}
  \qquad\=\kill
  \> $\fctx{\Gamma}{c}
           {\fall{t :\oty}
                 {\fimp{\fatm{\Gamma}{t : \tpty}}
                   {\fatm{\emptyce}{t:\tpty}}}}$.
\end{tabbing}
should hold.
%
We can in fact easily show this formula to be valid by
using Theorem~\ref{th:atomictype} and an induction on the height of
the derivation for $\fatm{G}{t : \tpty}$ for a closed term $t$ and a
closed instance $G$ of $c$.
%
Using the validity of this formula, we can also easily argue that the
following formula that expresses a strengthening property pertaining to
the equality of types is also valid:
\begin{tabbing}
  \ \ \=\kill
  \> $\fctx{\Gamma}{c}
           {\fall{d :\oty}
           {\fall{t_1 :\oty}
           {\fall{t_2 : \oty}
                 {\fimp{\fatm{\Gamma}{d : \eqty\app t_1\app t_2}}
                 {\fatm{\emptyce}{d:\eqty\app t_1\app t_2}}}}}}$.
\end{tabbing}

The property of uniqueness of type assignments for the STLC can be expressed through the
following formula:
\begin{tabbing}
\qquad\=\qquad\=\qquad\=\qquad\=\kill
\>$\fctx{\Gamma}{c}{\fall{e:\oty}{\fall{t_1:\oty}{\fall{t_2:\oty}{\fall{d_1:\oty}{\fall{d_2:\oty}{}}}}}}$\\
\>\>$\fimp{\fatm{\Gamma}{e:\tmty}}
          {\fimp{\fatm{\Gamma}{t_1:\tpty}}
                {\fimp{\fatm{\Gamma}{t_2:\tpty}}{}}}$\\
\>\>\> $\fimp{\fatm{\Gamma}{d_1:\ofty\app e\app t_1}}
             {\fimp{\fatm{\Gamma}{d_2:\ofty\app e\app t_2}}{}}$\\
\>\>\>\> $\fexists{d_3:\oty}{\fatm{.}{d_3:\eqty\app t_1\app t_2}}$.
\end{tabbing}
%
This formula can be seen to be valid using the strengthening property
just described if we can establish the validity of the formula
\begin{tabbing}
\qquad\=\qquad\=\qquad\=\qquad\=\kill
\>$\fctx{\Gamma}{c}{\fall{e:\oty}{\fall{t_1:\oty}{\fall{t_2:\oty}{\fall{d_1:\oty}{\fall{d_2:\oty}{}}}}}}$\\
\>\>$\fimp{\fatm{\Gamma}{e:\tmty}}
          {\fimp{\fatm{\Gamma}{t_1:\tpty}}
                {\fimp{\fatm{\Gamma}{t_2:\tpty}}{}}}$\\
\>\>\> $\fimp{\fatm{\Gamma}{d_1:\ofty\app e\app t_1}}
             {\fimp{\fatm{\Gamma}{d_2:\ofty\app e\app t_2}}{}}$\\
\>\>\>\> $\fexists{d_3:\oty}{\fatm{\Gamma}{d_3:\eqty\app t_1\app t_2}}$.
\end{tabbing}
To show this, it suffices to argue that, for a closed context
expression $G$ that instantiates the schema $c$ and for closed
expressions $d_1$, $d_2$, $e$, $t_1$, and $t_2$, if the formulas
$\fatm{G}{e:\tmty}$, $\fatm{G}{t_1:\tpty}$, 
$\fatm{G}{t_2:\tpty}$, $\fatm{G}{d_1 : \ofty\app e\app t_1}$ and 
$\fatm{G}{d_2 : \ofty\app e\app t_2}$ are valid, then there must be a
closed expression $d_3$ such that
$\fatm{G}{d_3 : \eqty\app t_1\app t_2}$ is also valid.
%
Such an argument can be constructed by induction on the height
of the LF derivation of $\lfchecktype{G}{d_1}{\ofty\app e\app t_1}$,
which we analyze using Theorem~\ref{th:atomictype} in the manner
discussed earlier. 
%
There are essentially four cases to consider, corresponding to
whether the head symbol of $d_1$ is \ofemptytm, \ofapptm, 
\oflamtm, or a nominal constant that is assigned the type
$(\ofty\app n\app t_1)$ in $G$ where $n$ is also a nominal constant that
is bound in $G$.
%
In the last case, we use the fact that the validity of
$\fatm{G}{d_1 : \ofty\app e\app t_1}$ implies that $\lfctx{G}$ is
derivable to conclude the uniqueness of $n$ and, hence, of the typing.
%
The argument when $d_1$ is \ofemptytm\ has an obvious form.
%
The argument when $d_1$ has \ofapptm\ or \oflamtm\ as its head symbol
will invoke the induction hypothesis. 
%
In the case where the head symbol is \oflamtm, we will need
to consider a shorter derivation of a typing judgement in which the
context has been enhanced.
%
However, we will be able to use the induction hypothesis by observing
that the enhancements to the context conform to the constraints
imposed by the context schema. 
%
Note that the form of $d_1$ also constrains the form of $e$ in all the
cases, a fact that is used implicitly in the analysis outlined.

\subsection{Nominal Constants and Invariance Under Permutations}
\label{ssec:permutations}

The particular choices for bound variable names in the kinds, types
and terms that comprise LF expressions are considered irrelevant.
%
This understanding is built in concretely through the notion of
$\alpha$-conversion that renders equal expressions that differ
only in the names used for such variables.
%
Typing derivations transform expressions with bound variables into
ones where variables are ostensibly free but in fact bound implicitly
in the associated contexts.
%
The lack of importance of name choices is reflected in this case in an
invariance in the validity of typing judgements under a suitable
renaming of variables appearing in the judgements.
%
In a situation where context variables are represented by nominal
constants, this property can be expressed via an invariance of formula
validity under permutations of nominal constants as we describe here.
%
We begin with a definition of the notions of permutations of
nominal constants and their applications to expressions. 
\begin{definition}%[Permutation]

A permutation of the nominal constants is an arity type preserving
bijection from $\noms$ to $\noms$ that differs from the identity
map at only a finite number of constants. 
%
The permutation that maps $n_1,\ldots, n_m$ to $n_1',\ldots,n_m'$,
respectively, and is the identity everywhere else is written as 
$\{n_1'/n_1,\ldots,n_m'/n_m\}$.
%
The support of a permutation $\pi=\{n_1'/n_1,\ldots,n_m'/n_m\}$, 
denoted by $\supp{\pi}$, is the collection of nominal constants
$\{n_1,\ldots, n_m\}$ or, identically, $\{n_1',\ldots,n_m'\}$.
%
Every permutation $\pi$ has an obvious inverse that is written as
$\inv{\pi}$. 
\end{definition}

\begin{definition}%[Permutation Application]
The application of a permutation $\pi$ to an expression $E$ of a
variety of kinds is described below and is denoted in all cases by
$\permute{\pi}{E}$.
%
If $E$ is a term, type, or kind then the
application consists of replacing each nominal constant $n$ that
appears in $E$ with $\pi(n)$. 
%
If $E$ is a context then the application of $\pi$ to $E$
replaces each explicit binding $n:A$ in $E$ with 
$\pi(n):\permute{\pi}{A}$.
%
If $E$ is an LF judgement $\mathcal{J}$ then the permutation is applied
to each component of the judgement in the way described above.
%
If $E$ is a formula then the permutation is applied to its component parts.
%
The application of $\pi$ to a term variables substitution
$\{\langle x_1,M_1,\alpha_1\rangle,\ldots,
   \langle x_n,M_n,\alpha_n\rangle\}$
yields the substitution 
$\{\langle x_1,\permute{\pi}{M_1},\alpha_1\rangle,\ldots,
   \langle x_n,\permute{\pi}{M_n},\alpha_n\rangle\}$.
%
The application of $\pi$ to a context variables substitution
$\{G_1/\Gamma_1,\ldots, G_n/\Gamma_n\}$ yields
$\{\permute{\pi}{G_1}/\Gamma_1,\ldots, \permute{\pi}{G_n}/\Gamma_n\}$.
\end{definition}

The following theorem expresses the property of interest
concerning LF judgements cast in the form relevant to \logic. 
%
\begin{theorem}\label{th:perm-lf}
Let LF judgements and derivations be recast in the form discussed
earlier in this section: variables that are bound in a context are
represented by nominal constants and the rules \canonkindpi,
\canonfampi\ and \canontermlam\ introduce fresh nominal constants into
contexts and replace variables in kinds, types and terms with these
constants.
%
In this context, let $\mathcal{J}$ be an LF judgement which has a
derivation. 
%
Then for any permutation $\pi$, $\permute{\pi}{\mathcal{J}}$ is derivable.
%
Moreover, the structure of this derivation is the same as that for
$\mathcal{J}$.
\end{theorem} 
\begin{proof}
This proof is by induction on the derivation for $\mathcal{J}$.
%
Perhaps the only observation worthy of note is that the freshness of
nominal constants used in \canonkindpi, \canonfampi, and
\canontermlam\ rules is preserved under permutations of nominal
constants. 
\end{proof}
%
The above observation underlies the main theorem of this section.
%
\begin{theorem}\label{th:perm-form}
Let $F$ be a closed formula and let $\pi$ be a permutation.
%
Then $F$ is valid if and only if $\permute{\pi}{F}$ is valid.
\end{theorem}
\begin{proof}
Noting that $\inv{\pi}$ is also a permutation and that
$\permute{\inv{\pi}}{\permute{\pi}{F}}$ is $F$, it suffices to prove
the claim in only one direction.
%
We do this by induction on the structure of $F$.

The desired result follows easily from
Theorem~\ref{th:perm-lf} and the relationship of validity to LF
derivability when $F$ is atomic.
%
The cases where $F$ is $\ftrue$ or $\ffalse$ are trivial and the
ones in which $F$ is $\fimp{F_1}{F_2}$, $\fand{F_1}{F_2}$ or
$\myfor{F_1}{F_2}$ are easily argued with recourse to the induction
hypothesis and by noting that the permutation distributes to the
component formulas.

In the case where $F$ is $\fctx{\Gamma}{\mathcal{C}}{F'}$, we first
note that if $\csinst{\noms}{\emptyset}{\mathcal{C}}{G}$ has a
derivation then
$\csinst{\noms}{\emptyset}{\mathcal{C}}{\permute{\inv{\pi}}{G}}$ must also
have one.
%
From this and the validity of $F$ it follows that
$\subst{\{(\permute{\inv{\pi}}{G})/\Gamma\}}{F'}$ must be valid.
%
Moreover, $\subst{\{(\permute{\inv{\pi}}{G})/\Gamma\}}{F'}$ has the same
structural complexity as $F'$.
%
Hence, by the induction hypothesis,
$\permute{\pi}{(\subst{\{\permute{(\inv{\pi}}{G})/\Gamma\}}{F'})}$ is valid.
%
Noting that this formula is the same as
$\subst{\{G/\Gamma\}}{(\permute{\pi}{F'})}$ and that
$\fctx{\Gamma}{\mathcal{C}}{\permute{\pi}{F'}}$ is identical to 
$\permute{\pi}{(\fctx{\Gamma}{\mathcal{C}}{F'})}$, the validity of
$\permute{\pi}{F}$ easily follows.

Suppose that $F$ has the form $\fall{x:\alpha}{F'}$.
%
We observe here that if
$\stlctyjudg{\noms \cup \STLCGamma_0}{M}{\alpha}$ has a derivation
then
$\stlctyjudg{\noms \cup \STLCGamma_0}{\permute{\inv{\pi}}{M}}{\alpha}$
has one too and that
$\permute{\pi}{(\hsubst{\{\langle x,\permute{\inv{\pi}}{M},\alpha\rangle\}}{F'})}$
is the same formula as
$\hsubst{\{\langle x,M,\alpha\rangle\}}{(\permute{\pi}{F'})}$.
%
Using the definition of validity, the induction hypothesis and the
fact that permutation distributes to the component formula together
with the above observations, we may easily conclude that
$\permute{\pi}{F}$ is valid.

Finally, suppose that $F$ is of the form $\fexists{x:\alpha}{F'}$.
%
Here we note that if
$\stlctyjudg{\noms \cup \STLCGamma_0}{M}{\alpha}$ has a derivation
then
$\stlctyjudg{\noms \cup \STLCGamma_0}{\permute{\pi}{M}}{\alpha}$
has one too and that
$\permute{\pi}{(\hsubst{\{\langle x, M,\alpha\rangle\}}{F'})}$
is the same formula as
$\hsubst{\{\langle
  x,\permute{\pi}{M},\alpha\rangle\}}{(\permute{\pi}{F'})}$.
%
Using the definition of validity and the induction hypothesis, it is
now easy to conclude that $\permute{\pi}{F}$ must be valid.
\end{proof}
