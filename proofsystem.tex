\section{A Proof System for the Logic}\label{sec:proofsystem}

\begin{figure*}
\begin{tabular}{c}
\infer
      {\seq{\mathbb{N}}
           {\Psi}
           {\Xi}
           {\Omega}
           {\mathcal{Q}_1.
             \fimp{F_1}{\fimp{\ldots}
                       {\mathcal{Q}_{k-1}.
                         \fimp{F_{k-1}}
                              {\mathcal{Q}_k.
                                  \fimp{\fatm{G}{M:A}}
                                        {\fimp{\ldots}{F_n}}}}}}}
      %% {\deduce{\mathcal{Q}_1.( 
      %%             \fimp{F_1}{\fimp{\ldots}
      %%                             {\mathcal{Q}_{k-1}}.(
      %%                               \fimp{F_{k-1}}
      %%                                    {\mathcal{Q}_k.
      %%                                       (\fimp{\eqannaux{i}{\fatm{G}{M:A}}}
      %%                                             {\fimp{\ldots}{F_n}})})})}
      %%         {\seq{\mathbb{N}}
      %%              {\Psi}
      %%              {\Xi}
      %%              {\Omega,
      %%                      \mathcal{Q}_1.( 
      %%                         \fimp{F_1}{\fimp{\ldots}
      %%                                         {\mathcal{Q}_{k-1}}.(
      %%                                          \fimp{F_{k-1}}
      %%                                               {\mathcal{Q}_k.
      %%                                                 (\fimp{\ltannaux{i}{\fatm{G}{M:A}}}
      %%                                                       {\fimp{\ldots}{F_n}})})})}
      %%              {}}}
              {\seq{\mathbb{N}}
                   {\Psi}
                   {\Xi}
                   {\Omega,
                           \mathcal{Q}_1. 
                              \fimp{F_1}{\fimp{\ldots}
                                              {\mathcal{Q}_{k-1}}.
                                               \fimp{F_{k-1}}
                                                    {\mathcal{Q}_k.
                                                      \fimp{\ltann{\fatm{G}{M:A}}}
                                                            {\fimp{\ldots}{F_n}}}}}
                   {\mathcal{Q}_1. 
                  \fimp{F_1}{\fimp{\ldots}
                                  {\mathcal{Q}_{k-1}}.
                                    \fimp{F_{k-1}}
                                         {\mathcal{Q}_k.
                                            \fimp{\eqann{\fatm{G}{M:A}}}
                                                  {\fimp{\ldots}{F_n}}}}}}
\end{tabular}
\caption{The Induction Rule}
\label{fig:induction}
\end{figure*}

We present here a high-level view of a proof system that we have developed
to provide a sound and effective means for mechanizing the kind of
validity arguments that we sketched in Section~\ref{ssec:logic-examples}.
%
We begin by explaining the structure of judgements considered by the
proof system before proceeding to a sketch of its inference rules.

\subsection{The Judgements for the Proof System}\label{ssec:sequents}

The proof system is oriented around deriving \emph{sequents} of the
form $\seq{\mathbb{N}}{\Psi}{\Xi}{\Omega}{F}$, where $\mathbb{N}$
finite set of nominal constants, $\Psi$ is a finite set of term
variables with associated arity types, $\Xi$ is a a finite set of
context variables with types of the kind described below, 
$\Omega$ is a finite set of \emph{assumption formulas}, and $F$ is a
\emph{conclusion} or \emph{goal} formula; 
here, $\mathbb{N}$ is the \emph{support set} of the
sequent, $\Psi$ is its \emph{eigenvariables context} and $\Xi$ is its
\emph{context variables context}.
%
The formulas in $\Omega \cup \{F\}$ must be formed out of the symbols
in $\mathbb{N}$, $\Psi$, $\Xi$ and the (implicit) signature $\Sigma$,
and they must be well-formed with respect to these collections in the
sense explained in Section~\ref{ssec:formulas}. 
%
The members of $\Xi$ have the form
$\ctxvarty{\Gamma}{\mathbb{N}_{\Gamma}}{\ctxty{\mathcal{C}}{G_1,\ldots,G_n}}$,
where $\mathbb{N}_{\Gamma} \subseteq \mathbb{N}$,
$\mathcal{C}$ is a context schema, and $G_1,\ldots,G_n$ is a listing of
instances of block schemas from $\mathcal{C}$ in which the types assigned to
nominal constants are well-formed with respect to the arity context
determined by $\Sigma$, $\mathbb{N} \setminus \mathbb{N}_\Gamma$, and $\Xi$. 
%
This ``typing'' of the variable $\Gamma$ is intended to limit its
range to closed contexts obtained by interspersing instances of block
schemas from $\mathcal{C}$ in which nominal constants from
$\mathbb{N}_\Gamma$ do not appear between instances of
$G_1,\ldots,G_n$ obtained by substituting terms for the variables in
$\Psi$ that are well-formed at the required arity types with respect
to the arity context determined by $\Sigma$ and the collection of
nominal constants not appearing in the support set of the
sequent.\footnote{Intuitively, the substitutions permitted for
  $\Gamma$ will be instances of the context schema $\mathcal{C}$ in
  which the nominal constants in $\mathbb{N}_\Gamma$ do not appear and
  that contain instances of $G_1,\ldots,G_n$ in the order listed.}

The basic notion of meaning for sequents is one that pertains to
closed sequents, i.e. ones of the form
$\seqsans{\mathbb{N}}{\Omega}{F}$. 
%
Such a sequent is \emph{valid} if $F$ is valid or one of the
assumption formulas in $\Omega$ is not valid.
%
A sequent of the general form
$\seq{\mathbb{N}}{\Psi}{\Xi}{\Omega}{F}$ is then considered valid if
all of its instances obtained by substituting closed terms not
containing the nominal constants in $\mathbb{N}$ and respecting arity
typing constraints for the variables in $\Psi$ and replacing the
variables in $\Xi$ with closed contexts respecting their types in the
manner described above are valid.
%
The goal of showing that a formula $F$ whose nominal
constants are contained in the set $\mathbb{N}$ is valid now reduces to
showing the validity of the sequent $\seq{\mathbb{N}}{\emptyset}{\emptyset}{\emptyset}{F}$.

\subsection{The Proof Rules}\label{ssec:proofrules}

The proof system comprises two kinds of rules: those that pertain
to the logical symbols and structural aspects of sequents and those
that encode the interpretation of atomic formulas as assertions of
derivability in LF.
%
In the first category are included the usual contraction and weakening
rules (that are also extended to treat the support set and the
eigenvariables and context variables contexts), and the cut rule that
facilitates the use of well-formed formulas as lemmas.
%
The logical rules also take the expected form with two noteworthy
aspects: in the statement of axioms, it is necessary to incorporate
the equivalence of atomic formulas under a suitable form of
permutation of nominal constants, and the eigenvariables that are
introduced for (essential) universal quantifiers must be
raised over the support set of the sequent to correctly reflect
dependencies in the context of our interpretation of
sequents~\cite{miller92jsc}. 

The rules in the second category constitute the truly novel part of
the proof system and they are discussed in more detail below.

\subsubsection{The Treatment of Atomic Formulas}

The calculus builds in the understanding of formulas of the form
$\fatm{G}{M : A}$ via LF derivability.
%
If $A$ is a type of the form $\typedpi{x}{A_1}{A_2}$, then $M$ must
have the form $\lflam{x}{M'}$ and the atomic formula can be replaced
by one of the form $\fatm{G,n:A_1}{M':A_2}$ in the sequent; here, $n$
must be a nominal constant not already in the support set and if $G$
contains a context variable then its type annotation must be changed
to prohibit the occurrence of $n$ in its instantiations.
%
If $A$ is an atomic type and $\fatm{G}{M : A}$ is the goal formula,
then the corresponding rule embodies the use of
Theorem~\ref{th:atomictype} in taking a step in the
validation of the typing judgement.
%
Note that if $G$ begins with a context variable $\Gamma$, then the
blocks in the ``type'' of $\Gamma$ must also be considered in
determining the type assignment to the head of $M$. 

When the formula $\fatm{G}{M : A}$ in which $A$ is an atomic type
appears as an assumption formula in the sequent, then the rule must
capture a \emph{case analysis} style of reasoning: 
we must identify all the possibilities for the valid
closed instances of this formula and analyze the validity of the
sequent based on these instances.
%
The difficulty, however, is that there may be far too many closed
instances to consider explicitly.
%
This issue can be refined into two specific problems that must be
addressed. 
%
First, the context $G$ might begin with a context variable and we must
then identify a realistic way to consider all the instantiations of that
variable that yield an actual, closed context.
%
Second, we must describe a manageable approach to considering all
possible instantiations for the term variables that may appear in
$\fatm{G}{M : A}$. 

The first problem is solved in the enunciation of the rule through an
\emph{incremental elaboration} of a context variable.
%
Suppose that $G$ begins with the context variable $\Gamma$
corresponding to which there is the declaration
$\ctxvarty{\Gamma}{\mathbb{N}_{\Gamma}}{\ctxty{\mathcal{C}}{G_1,\ldots,G_k}}$
in the context variables context.
%
We would at the outset need to consider all the constants in $\Sigma$
and all the nominal constants identified explicitly in $G$, which
includes the ones declared in $G_1,\ldots,G_k$, as potential heads for
$M$ in the formula $\fatm{G}{M:A}$.
%
Additionally, this head may come from a part of $\Gamma$ that has not
yet been made explicit.
%
To account for this, the rule considers all the possible instances for
the block declarations constituting $\mathcal{C}$ and all possible
locations for such blocks in the sequence $G_1,\ldots,G_k$.
%
We note that the invariance of atomic formulas under permutations of
nominal constants allows the number of such instances that have to be
examined to be kept finite and, in practice, small.

The second problem is addressed by first describing a notion
of unification that will ensure that all closed instances will be
considered and then identifying the idea of a \emph{covering set of
unifiers} that obviates an exhaustive consideration.
%
To elaborate, suppose that the (nominal) constant $h$ with
type $\typedpi{x_1}{A_1}{\ldots\typedpi{x_n}{A_n}{A'}}$ has been
identified  as the candidate head for $M$.
%
For $1\leq i \leq n$, let $t_i$ be terms representing fresh
variables raised over the support set of the sequent.\footnote{If
  $n_1,\ldots,n_{\ell}$ is a listing of $\mathbb{N}$ then, for $1 \leq
  i \leq n$, $t_i$ is the term $(z_i\app n_1\app \cdots\app n_{\ell})$
  where $z_i$ is a fresh variable of suitable arity type.}
%
Then, based on the notion of unification described, for each
substitution $\theta$ that unifies 
$(h\app t_1\app\ldots\app t_n)$ and $M$ on the one hand and $A$ and
$\hsubst{\{\langle x_1,t_1,\erase{A_1} \rangle,
  \ldots,\langle x_n,t_n,\erase{A_n}\rangle\}}{A'}$ on the other, 
it suffices to consider the derivability of the sequent that results
from replacing $\fatm{G}{M:A}$ with the set of
formulas 
\begin{tabbing}
\quad\=\qquad\qquad\qquad\qquad\qquad\=\kill
\>$\{\fatm{G}{t_i : \hsubst{\{\langle x_1,t_1,\erase{A_1} \rangle,
  \ldots,\langle
  x_{i-1},t_{i-1},\erase{A_{i-1}}\rangle\}}{A_i}}\ \vert$\\
\>\>$1 \leq i \leq n\}$
\end{tabbing}
and then applying the substitution $\theta$.
%
However, this will still result in a a large number of cases since the
collection of unifiers must include all relevant closed instances for
the analysis to be sound.
%
The notion of a covering set provides a means for limiting attention
to a small subset of unifiers while still preserving soundness.

\subsubsection{Induction over the Heights of LF Derivations}

The ability to reason inductively over the heights of LF typing
derivations is realized in the proof system by using an
annotation based scheme inspired by Abella~\cite{baelde14jfr,
  gacek09phd}.
%
Specifically, we add to the syntax two additional forms of atomic
formulas: $\eqann{\fatm{G}{M:A}}$ and
$\ltann{\fatm{G}{M:A}}$.
%
These represent, respectively, a formula that has an LF derivation of
some given height and another formula of strictly smaller height; the
latter formula is obtained typically by an unfolding step embodied in
the use of a case analysis rule.
% 
The induction rule then has the form shown in Figure~\ref{fig:induction}
where $\mathcal{Q}_i$ represent a sequence of context quantifiers or universal
term quantifiers and the annotations $@$ and $*$ must not already appear
in the conclusion sequent.\footnote{Nested induction can be
  accommodated by generalizing the annotations to the form $@^i$ and
  $*^i$ for any natural number $i$.}
%
The premise of this proof rule can be viewed as providing a proof schema for
constructing an argument of validity for any particular height $m$, and so by 
an inductive argument we can conclude that the formula will be valid 
regardless of the derivation height.
%
For this proof rule to be useful in reasoning, the case analysis rule
must be generalized to permit a movement from a formula
annotated with $@$ to ones annotated by $*$ when reduced, an ability
justified by Theorem~\ref{th:atomictype}. 

\subsubsection{Rules Encoding LF Metatheorems}

LF admit several metatheorems such as weakening, strengthening,
permutation and substitution, that are useful in
reasoning about specifications.
%
These metatheorems are built into the proof system via (sound) axioms.
%
For example, (one version of) the strengthening metatheorem is encoded
in the axiom
\[
\infer
      {\seq{\mathbb{N}}{\Psi}{\Xi}{\Omega}{\fimp{\fatm{G,n:B}{M:A}}{\fatm{G}{M:A}}}}
      {\begin{array}{c}
         n\mbox{ does not appear in }M\mbox{, } A\mbox{, or the explicit bindings in }G
%         \\
%        Ann\in\{\mbox{None}, @^i,*^i\}
       \end{array}}
\]
%
These axioms can be combined with the cut rule to encode the informal reasoning process. 
