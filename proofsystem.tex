\section{A Proof System for the Logic}\label{sec:proofsystem}

We now describe a sequent calculus that supports arguments of validity
of the kind outlined in Section~\ref{ssec:example}.
%
We aim only to present the spirit of this calculus, leaving
its detailed consideration again to other
work~\cite{nadathur21arxiv,southern21phd}.
%
We note also that the main goal for the calculus is to provide
a means for sound and effective reasoning rather than to be complete
with respect to the semantics described for \logic.
%
We begin by explaining the structure of sequents before proceeding to
a discussion of the inference rules.

\subsection{The Structure of Sequents}\label{ssec:sequents}

A sequent, written as $\seq{\mathbb{N}}{\Psi}{\Xi}{\Omega}{F}$, is a
judgement that relates a finite subset $\mathbb{N}$ of nominal
constants with associated arity types, a finite set $\Psi$ of term
variables also with associated arity types, a finite set $\Xi$ of
context variables with types of the kind described below, a finite set
$\Omega$ of \emph{assumption formulas} and a \emph{conclusion} or
\emph{goal} formula $F$; 
here, $\mathbb{N}$ is the \emph{support set} of the
sequent, $\Psi$ is its \emph{eigenvariables context} and $\Xi$ is its
\emph{context variables context}.
%
The formulas in $\Omega \cup \{F\}$ must be formed out of the symbols
in $\mathbb{N}$, $\Psi$, $\Xi$ and the (implicit) signature $\Sigma$,
and they must be well-formed with respect to these collections in the
sense explained in Section~\ref{ssec:formulas}. 
%
The members of $\Xi$ have the form
$\ctxvarty{\Gamma}{\mathbb{N}_{\Gamma}}{\ctxty{\mathcal{C}}{G_1,\ldots,G_n}}$,
where $\mathbb{N}_{\Gamma}$ is a collection of nominal constants,
$\mathcal{C}$ is a context schema, and $G_1,\ldots,G_n$ is a listing of
instances of block schemas from $\mathcal{C}$ in which the types assigned to
nominal constants are well-formed with respect to $\Sigma$,
$\mathbb{N} \setminus \mathbb{N}_\Gamma$, and $\Xi$. 
%
This ``typing'' of the variable $\Gamma$ is intended to limit its
range to closed contexts obtained by interspersing instances of block
schemas from $\mathcal{C}$ in which nominal constants from
$\mathbb{N}_\Gamma$ do not appear between instances of
$G_1,\ldots,G_n$ obtained by substituting terms formed from $\Sigma$
and nominal constants not appearing in the support set of the
sequent for the variables in $\Psi$; the substitutions for the
variables in $\Psi$ must respect arity typing and the LF types in the
resulting context must be well-typed in an arity sense. 

The basic notion of meaning for sequents is one that pertains to
closed sequents, i.e. ones of the form
$\seqsans{\mathbb{N}}{\Omega}{F}$. 
%
Such a sequent is \emph{valid} if $F$ is valid or one of the
assumption formulas in $\Omega$ is not valid.
%
A sequent of the general form
$\seq{\mathbb{N}}{\Psi}{\Xi}{\Omega}{F}$ is then considered valid if
all of its instances obtained by substituting closed terms not
containing the nominal constants in $\mathbb{N}$ and respecting arity
typing constraints for the variables in $\Psi$ and replacing the
variables in $\Xi$ with closed contexts respecting their types in the
manner described above are valid.
%
The goal of showing that a formula $F$ whose nominal
constants are contained in the set $\mathbb{N}$ is valid now reduces to
showing the validity of the sequent $\seq{\mathbb{N}}{\emptyset}{\emptyset}{\emptyset}{F}$.

\subsection{The Rules for Deriving Sequents}\label{ssec:proofrules}

The sequent calculus comprises two kinds of rules: those that pertain
to the logical symbols and structural aspects of sequents and those
that encode the interpretation of atomic formulas as assertions of
derivability in LF. We discuss the rules under these categories below,
focusing mainly on the latter kind of rules. For paucity of space, we
do not present the rules explicitly but rather discuss their intuitive
content. We also note that all the rules that we describe have been
%GN: leaving out the thesis reference here because some of the proofs
%there have gaps and we don't want people to throw that at us.
shown to be sound~\cite{nadathur21arxiv}.

\subsubsection{Structural and Logical Rules}

The calculus includes the usual contraction and weakening rules
pertaining to assumption formulas.
%
Also included are rules for adding and removing entries from the
support set and the eigenvariables and context variables contexts when
these additions do not impact the overall well-formedness of
sequents.
%
Finally, the cut rule, which facilitates the use of well-formed
formulas as lemmas, is also present in the collection.

The most basic logical rule is that of an axiom.
%
The main deviation from the usual form for this is the
incorporation of the invariance of LF derivability under permutations
of the names of context variables; this is realized in our proof
system via a notion of equivalence of formulas under permutations of nominal
constants.   
%
The rules for the connectives and quantifiers take the expected form.
%
The only significant point to note is that eigenvariables that are
introduced for (essential) universal quantifiers must be
raised over the support set of the sequent to correctly reflect
dependencies given our interpretation of sequents~\cite{miller92jsc}.

\subsubsection{The Treatment of Atomic Formulas}

The calculus builds in the understanding of formulas of the form
$\fatm{G}{M : A}$ via LF derivability.
%
If $A$ is a type of the form $\typedpi{x}{A_1}{A_2}$, then $M$ must
have the form $\lflam{x}{M'}$ and the atomic formula can be replaced
by one of the form $\fatm{G,n:A_1}{M':A_2}$ in the sequent; here, $n$
must be a nominal constant not already in the support set and if $G$
contains a context variable then its type annotation must be changed
to prohibit the occurrence of $n$ in its instantiations.
%
If $A$ is an atomic type and $\fatm{G}{M : A}$ is the goal formula,
then the corresponding rule allows a step to be taken in the
validation of the typing judgement; specifically, if $M$ is the term
$(h\app M_1\app\ldots\app M_n)$ where $h$ is a constant or a nominal
constant to which $\Sigma$ or $G$ assigns the
LF type $\typedpi{x_1}{A_1}{\ldots\typedpi{x_n}{A_n}{A'}}$ and $A$ is
identical to $\hsubst{\{\langle x_1,M_1\erase{A_1} \rangle,
  \ldots,\langle x_n,M_n,\erase{A_n}\rangle\}}{A'}$, then the rule
leads to the consideration of the derivation of sequents in which
the goal formula is changed to $\fatm{G}{\hsubst{\{\langle x_1,M_1\erase{A_1} \rangle,
  \ldots,\langle x_{i-1},M_{i-1},\erase{A_{i-1}}\rangle\}}{M_i}}$ for
$1 \leq i \leq n$. Note that if $G$ begins with a context variable $\Gamma$,
  then the assignments in the blocks in the ``type'' of $\Gamma$ are
  considered to be assignments in $G$.

The most contentful part of the treatment of atomic formulas
is when the formula $\fatm{G}{M : A}$ in which $A$ is an atomic type
appears as an assumption formula in the sequent. 
%
The example in Section~\ref{ssec:example} demonstrates the
\emph{case analysis} style of reasoning that we would want to capture 
in the proof rule; we must identify all the possibilities for the valid
closed instances of this formula and analyze the validity of the
sequent based on these instances.
%
The difficulty, however, is that there may be far too many closed
instances to consider explicitly.
%
This issue can be refined into two specific problems that must be
addressed. 
%
First, the context $G$ might begin with a context variable and we must
then identify a realistic way to consider all the instantiations of that
variable that yield an actual, closed context.
%
Second, we must describe a manageable approach to considering all
possible instantiations for the term variables that may appear in
$\fatm{G}{M : A}$. 

The first problem is solved in the enunciation of the rule through an
\emph{incremental elaboration} of a context variable that is driven by
the atomic formula under scrutiny.
%
Suppose that $G$ begins with the context variable $\Gamma$
corresponding to which there is the declaration
$\ctxvarty{\Gamma}{\mathbb{N}_{\Gamma}}{\ctxty{\mathcal{C}}{G_1,\ldots,G_k}}$
in the context variables context.
%
We would at the outset need to consider all the constants in $\Sigma$
and all the nominal constants identified explicitly in $G$, which
includes the ones declared in $G_1,\ldots,G_k$, as potential heads for
$M$ in the formula $\fatm{G}{M:A}$.
%
Additionally, this head may come from a part of $\Gamma$ that has not
yet been made explicit.
%
To account for this, the rule considers all the possible instances for
the block declarations constituting $\mathcal{C}$ and all possible
locations for such blocks in the sequence $G_1,\ldots,G_k$.
%
We note that the number of such instances that have to be examined 
is finite because it suffices to consider exactly one representative
for any nominal constant that does not appear in the support set of
the sequent; this observation follows from the invariance of LF typing
judgements under permutations of names for the variables bound in the
context. 

The second problem is addressed by first describing a notion
of unification that will ensure that all closed instances will be
considered and then identifying the idea of a \emph{covering set of
unifiers} that enables us to avoid an exhaustive consideration.
%
To elaborate a little on this approach, suppose that the (nominal)
constant $h$ with LF type
$\typedpi{x_1}{A_1}{\ldots\typedpi{x_n}{A_n}{A'}}$ has been identified 
as the candidate head for $M$.
%
Further, for $1\leq i \leq n$, let $t_i$ be terms representing fresh
variables raised over the support set of the sequent.\footnote{If
  $n_1,\ldots,n_{\ell}$ is a listing of $\mathbb{N}$ then, for $1 \leq
  i \leq n$, $t_i$ is the term $(z_i\app n_1\app \cdots\app n_{\ell})$
  where $z_i$ is a fresh variable of suitable arity type.}
%
Then, based on the notion of unification described, for each
substitution $\theta$ that unifies 
$(h\app t_1\app\ldots\app t_n)$ and $M$ on the one hand and $A$ and
$\hsubst{\{\langle x_1,t_1\erase{A_1} \rangle,
  \ldots,\langle x_n,t_n,\erase{A_n}\rangle\}}{A'}$ on the other, 
it suffices to consider the derivability of the sequent that results
from replacing $\fatm{G}{M:A}$ in the original sequent with the set of
formulas 
\[
  \{\fatm{G}{t_i : \hsubst{\{\langle x_1,t_1\erase{A_1} \rangle,
  \ldots,\langle
  x_{i-1},t_{i-1},\erase{A_{i-1}}\rangle\}}{A_i}}\ \vert\ 1 \leq i \leq
  n\}
\]
and then applying the substitution $\theta$.
%
However, this will still result in a a large number of cases since the
collection of unifiers must include all relevant closed instances for
the analysis to be sound.
%
The notion of a covering set provides a means for limiting attention
to a small subset of unifiers while still preserving soundness.

\subsubsection{Induction over the Heights of LF Derivations}

The example in Subsection~\ref{ssec:example} also illustrates the role
of induction over the heights of LF typing derivations in informal
reasoning. 
%
This kind of induction is realized in our sequent calculus by using an
annotation based scheme inspired by Abella~\cite{baelde14jfr,
  gacek09phd}.
%
Specifically, we add to the syntax two additional forms of atomic
formulas: $\eqannaux{i}{\fatm{G}{M:A}}$ and
$\ltannaux{i}{\fatm{G}{M:A}}$.
%
These represent, respectively, a formula that has an LF derivation of
some given height and another formula of strictly smaller height; the
latter formula is obtained typically by an unfolding step embodied in
the use of a case analysis rule.
% 
The index $i$ on the annotation symbol is used to identify distinct pairs of 
$@$ and $*$ annotations.
%
The induction proof rule then has the form

\[
\infer[induction]
      {\seq{\mathbb{N}}
           {\Psi}
           {\Xi}
           {\Omega}
           {\mathcal{Q}_1.(
             \fimp{F_1}{\fimp{\ldots}
                       {\mathcal{Q}_{k-1}.(
                         \fimp{F_{k-1}}
                              {\mathcal{Q}_k.
                                  (\fimp{\fatm{G}{M:A}}
                                        {\fimp{\ldots}{F_n}})})}})}}
      {\deduce{\mathcal{Q}_1.( 
                  \fimp{F_1}{\fimp{\ldots}
                                  {\mathcal{Q}_{k-1}}.(
                                    \fimp{F_{k-1}}
                                         {\mathcal{Q}_k.
                                            (\fimp{\eqannaux{i}{\fatm{G}{M:A}}}
                                                  {\fimp{\ldots}{F_n}})})})}
              {\seq{\mathbb{N}}
                   {\Psi}
                   {\Xi}
                   {\Omega,
                           \mathcal{Q}_1.( 
                              \fimp{F_1}{\fimp{\ldots}
                                              {\mathcal{Q}_{k-1}}.(
                                               \fimp{F_{k-1}}
                                                    {\mathcal{Q}_k.
                                                      (\fimp{\ltannaux{i}{\fatm{G}{M:A}}}
                                                            {\fimp{\ldots}{F_n}})})})}
                   {}}}
\]
where $\mathcal{Q}_i$ represent a sequence of context quantifiers or universal
term quantifiers and the annotations $@^i$ and $*^i$ must not already appear
in the conclusion sequent.
%
The premise of this proof rule can be viewed as providing a proof schema for
constructing an argument of validity for any particular height $m$, and so by 
an inductive argument we can conclude that the formula will be valid 
regardless of the derivation height.
%
This idea is made precise in a proof of soundness for the
rule~\cite{nadathur21arxiv,southern21phd}.

For this proof rule to be useful in reasoning, we will need a form of
case analysis which permits us to move from a formula annotated with
$@$ to one annotated by $*$ when reduced. 
%
Such a mechanism is built into the proof system and is used in the
examples in the next section.

\subsubsection{Rules Encoding Metatheorems Concerning LF Derivability}

Typing judgements in LF admit several metatheorems that are useful in
reasoning about specifications: if $\lfchecktype{\Gamma}{M}{A}$ has a
derivation then so does $\lfchecktype{\Gamma, x : A'}{M}{A}$ for a fresh
variable $x$ and any well-formed type $A'$ (weakening); if
$\lfchecktype{\Gamma, x : A'}{M}{A}$ has a derivation and $x$ does not
appear in $M$ or $A$ then so also does $\lfchecktype{\Gamma}{M}{A}$
(strengthening); if $\lfchecktype{\Gamma_1, x_1 : A_1, x_2 : A_2, \Gamma_2}{M}{A}$ has
a derivation and $x_1$ does not appear in $A_2$ then 
$\lfchecktype{\Gamma_1, x_2 : A_1, x_1 : A_1, \Gamma_2}{M}{A}$ must have a derivation
(permutation); and if $\lfchecktype{\Gamma_1}{M'}{A'}$ and
$\lfchecktype{\Gamma_1,x:A',\Gamma_2}{M}{A}$ have derivations then there
must be a derivation for
$\lfchecktype{\Gamma_1,\hsubst{\{\langle x,M',\erase{A'}\rangle\}}{\Gamma_2}}
         {\hsubst{\{\langle x,M',\erase{A'}\rangle\}}{M}}
         {\hsubst{\{\langle x,M',\erase{A'}\rangle\}}{A}}$ (substitution). 
%
Moreover, in the first three cases, the derivations are structurally
similar, e.g. they have the same heights.
%
These metatheorems are built into the sequent calculus via (sound)
axioms.
%
For example, (one version of) the strengthening metatheorem is encoded
in the axiom
\[
\infer[\lfstr]
      {\seq{\mathbb{N}}{\Psi}{\Xi}{\Omega}{\fimp{\fatm{G,n:B}{M:A}}{\fatm{G}{M:A}}}}
      {\begin{array}{c}
         n\mbox{ does not appear in }M\mbox{, } A\mbox{, or the explicit bindings in }G
%         \\
%        Ann\in\{\mbox{None}, @^i,*^i\}
       \end{array}}
\]
%
These axioms can be combined with the cut rule to encode the informal reasoning process. 
