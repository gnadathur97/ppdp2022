\section{Related Work and Conclusion}\label{sec:conclusion}

We have presented a logic in this paper that provides
a means for formalizing properties of LF specifications and have
sketched a proof system that supports the mechanization of arguments
of validity relative to it. 
%
While space does not permit the illustration of the proof
system, such examples may be found in~\cite{nadathur21arxiv} and
\cite{southern21phd}; in particular, Section 5
of~\cite{nadathur21arxiv} shows an encoding of the informal proof of
type uniqueness for the STLC in the proof system.
%
%% In complementary work, we have implemented the proof system in a proof
%% assistant called Adelfa~\cite{southern21phd,southern21lfmtp}. 

There have been other efforts towards building a capability for
reasoning about LF specifications albeit with significantly different
philosophical underpinnings. 
%
One prominent approach is that reflected in what might be called the
``Twelf family'' of systems.
%
The first realization of this approach is the Twelf system
itself~\cite{pfenning99cade}. 
%
This system uses LF once again to formalize properties of systems
described in it.
%
More specifically, the properties that are of interest are described
by other LF types that have a functional structure, identified by mode
declarations. 
%
The validity of these properties is then demonstrated by constructing
inhabitants of the ``output'' types for every possible value for the
``input'' types.
%
This approach has achieved much success even though it requires
properties that need to be established to be transformed into formulas
that have a $\forall\exists$ quantifier structure.
%
A drawback of the approach is that is that there isn't an
explicit proof to be extracted at the end of a development.
%
The logic $M_2^+$ has been enunciated towards mitigating this
issue~\cite{schurmann00phd}, and it appears possible to mechanically 
relate the ``reasoning'' embodied in a Twelf development to a proof in
this logic~\cite{wang13lfmtp}.
%
While some of the proof rules in $M_2^+$ bear a resemblance to the
ones in the proof system for \logic, there are significant differences
in the specifics of, for example, the treatment of case analysis and
induction.

Unlike \logic, the Twelf system and the $M_2^+$ logic do not
provide an explicit means for quantifying over contexts. It is
possible to parameterize a development by a context description, but
it is one fixed context that then permeates the development.
%
As an example, it is not possible to express the strengthening lemma 
pertaining to the equality of types in the STLC that we discussed in
Section~\ref{ssec:logic-examples}.
%
The Beluga system~\cite{pientka10ijcar} alleviates this problem by using
a richer version of type theory that allows for an explicit treatment
of contexts as its basis~\cite{nanevski08tocl}.
%
Beluga is based on a computational view of reasoning that is similar
in many respects to the philosophy underlying Twelf: one writes
dependently-typed recursive functions and the type system ensures that
the admitted 
functions are ones that are total and hence embody ``proofs'' of the
properties expressed by the types. 
%
This system is more expressive than Twelf, but the structure of the
formulas that encode properties is similarly limited. 
%
A recent development that appears related to this line of work is that
of {\sc Cocon}, a Martin-L\"{o}f style type theory that embeds LF within
a rich dependently typed calculus that supports
recursion~\cite{pientka2019arxiv}. 
%
It would be interesting to compare the reasoning capabilities that
result from this kind of a combination with what is possible to
achieve with an approach like ours.
%
We plan to explore this issue in future work.

Another approach that has been explored for reasoning about LF
specifications is based on their translation to a predicate logic
form.
%
A particular exemplar of this approach is one that translates LF
specifications into specifications in the logic of hereditary Harrop 
formulas~\cite{miller12proghol}, to then be reasoned about using the
Abella system~\cite{southern14fsttcs}.
% 
This approach allows benefit to be derived from 
the (generic) reasoning capabilities that have been developed for the
host system. 
%
In the mentioned example, several such advantages are derived from the
expressiveness of the logic underlying Abella~\cite{gacek11ic}: 
it is possible to define relations between contexts, to treat
binding notions explicitly in the reasoning process through the
$\nabla$-quantifier, and to use inductive (and co-inductive)
definitions in the reasoning logic.
%
There are, however, some drawbacks to the translation-based
approach. 
%
Perhaps the most significant problem is that the proof steps that can
be taken under it are determined by the logic of the host system, and
this may allow for more possibilities than are sensible in the LF
context.  
%
One way to overcome this difficulty is to design macro proof steps
that capture the natural process of reasoning about LF
specifications. 
%
In this respect, one important outcome of the work that we have
described here, especially for the structure that we have developed
for case analysis, might be an understanding of how one might
build within proof assistants such as Abella a targeted
capability for reasoning about LF specifications. 
